% Options for packages loaded elsewhere
\PassOptionsToPackage{unicode}{hyperref}
\PassOptionsToPackage{hyphens}{url}
\PassOptionsToPackage{dvipsnames,svgnames,x11names}{xcolor}
%
\documentclass[
  letterpaper,
  DIV=11,
  numbers=noendperiod]{scrartcl}

\usepackage{amsmath,amssymb}
\usepackage{iftex}
\ifPDFTeX
  \usepackage[T1]{fontenc}
  \usepackage[utf8]{inputenc}
  \usepackage{textcomp} % provide euro and other symbols
\else % if luatex or xetex
  \usepackage{unicode-math}
  \defaultfontfeatures{Scale=MatchLowercase}
  \defaultfontfeatures[\rmfamily]{Ligatures=TeX,Scale=1}
\fi
\usepackage{lmodern}
\ifPDFTeX\else  
    % xetex/luatex font selection
\fi
% Use upquote if available, for straight quotes in verbatim environments
\IfFileExists{upquote.sty}{\usepackage{upquote}}{}
\IfFileExists{microtype.sty}{% use microtype if available
  \usepackage[]{microtype}
  \UseMicrotypeSet[protrusion]{basicmath} % disable protrusion for tt fonts
}{}
\makeatletter
\@ifundefined{KOMAClassName}{% if non-KOMA class
  \IfFileExists{parskip.sty}{%
    \usepackage{parskip}
  }{% else
    \setlength{\parindent}{0pt}
    \setlength{\parskip}{6pt plus 2pt minus 1pt}}
}{% if KOMA class
  \KOMAoptions{parskip=half}}
\makeatother
\usepackage{xcolor}
\setlength{\emergencystretch}{3em} % prevent overfull lines
\setcounter{secnumdepth}{-\maxdimen} % remove section numbering
% Make \paragraph and \subparagraph free-standing
\makeatletter
\ifx\paragraph\undefined\else
  \let\oldparagraph\paragraph
  \renewcommand{\paragraph}{
    \@ifstar
      \xxxParagraphStar
      \xxxParagraphNoStar
  }
  \newcommand{\xxxParagraphStar}[1]{\oldparagraph*{#1}\mbox{}}
  \newcommand{\xxxParagraphNoStar}[1]{\oldparagraph{#1}\mbox{}}
\fi
\ifx\subparagraph\undefined\else
  \let\oldsubparagraph\subparagraph
  \renewcommand{\subparagraph}{
    \@ifstar
      \xxxSubParagraphStar
      \xxxSubParagraphNoStar
  }
  \newcommand{\xxxSubParagraphStar}[1]{\oldsubparagraph*{#1}\mbox{}}
  \newcommand{\xxxSubParagraphNoStar}[1]{\oldsubparagraph{#1}\mbox{}}
\fi
\makeatother

\usepackage{color}
\usepackage{fancyvrb}
\newcommand{\VerbBar}{|}
\newcommand{\VERB}{\Verb[commandchars=\\\{\}]}
\DefineVerbatimEnvironment{Highlighting}{Verbatim}{commandchars=\\\{\}}
% Add ',fontsize=\small' for more characters per line
\usepackage{framed}
\definecolor{shadecolor}{RGB}{241,243,245}
\newenvironment{Shaded}{\begin{snugshade}}{\end{snugshade}}
\newcommand{\AlertTok}[1]{\textcolor[rgb]{0.68,0.00,0.00}{#1}}
\newcommand{\AnnotationTok}[1]{\textcolor[rgb]{0.37,0.37,0.37}{#1}}
\newcommand{\AttributeTok}[1]{\textcolor[rgb]{0.40,0.45,0.13}{#1}}
\newcommand{\BaseNTok}[1]{\textcolor[rgb]{0.68,0.00,0.00}{#1}}
\newcommand{\BuiltInTok}[1]{\textcolor[rgb]{0.00,0.23,0.31}{#1}}
\newcommand{\CharTok}[1]{\textcolor[rgb]{0.13,0.47,0.30}{#1}}
\newcommand{\CommentTok}[1]{\textcolor[rgb]{0.37,0.37,0.37}{#1}}
\newcommand{\CommentVarTok}[1]{\textcolor[rgb]{0.37,0.37,0.37}{\textit{#1}}}
\newcommand{\ConstantTok}[1]{\textcolor[rgb]{0.56,0.35,0.01}{#1}}
\newcommand{\ControlFlowTok}[1]{\textcolor[rgb]{0.00,0.23,0.31}{\textbf{#1}}}
\newcommand{\DataTypeTok}[1]{\textcolor[rgb]{0.68,0.00,0.00}{#1}}
\newcommand{\DecValTok}[1]{\textcolor[rgb]{0.68,0.00,0.00}{#1}}
\newcommand{\DocumentationTok}[1]{\textcolor[rgb]{0.37,0.37,0.37}{\textit{#1}}}
\newcommand{\ErrorTok}[1]{\textcolor[rgb]{0.68,0.00,0.00}{#1}}
\newcommand{\ExtensionTok}[1]{\textcolor[rgb]{0.00,0.23,0.31}{#1}}
\newcommand{\FloatTok}[1]{\textcolor[rgb]{0.68,0.00,0.00}{#1}}
\newcommand{\FunctionTok}[1]{\textcolor[rgb]{0.28,0.35,0.67}{#1}}
\newcommand{\ImportTok}[1]{\textcolor[rgb]{0.00,0.46,0.62}{#1}}
\newcommand{\InformationTok}[1]{\textcolor[rgb]{0.37,0.37,0.37}{#1}}
\newcommand{\KeywordTok}[1]{\textcolor[rgb]{0.00,0.23,0.31}{\textbf{#1}}}
\newcommand{\NormalTok}[1]{\textcolor[rgb]{0.00,0.23,0.31}{#1}}
\newcommand{\OperatorTok}[1]{\textcolor[rgb]{0.37,0.37,0.37}{#1}}
\newcommand{\OtherTok}[1]{\textcolor[rgb]{0.00,0.23,0.31}{#1}}
\newcommand{\PreprocessorTok}[1]{\textcolor[rgb]{0.68,0.00,0.00}{#1}}
\newcommand{\RegionMarkerTok}[1]{\textcolor[rgb]{0.00,0.23,0.31}{#1}}
\newcommand{\SpecialCharTok}[1]{\textcolor[rgb]{0.37,0.37,0.37}{#1}}
\newcommand{\SpecialStringTok}[1]{\textcolor[rgb]{0.13,0.47,0.30}{#1}}
\newcommand{\StringTok}[1]{\textcolor[rgb]{0.13,0.47,0.30}{#1}}
\newcommand{\VariableTok}[1]{\textcolor[rgb]{0.07,0.07,0.07}{#1}}
\newcommand{\VerbatimStringTok}[1]{\textcolor[rgb]{0.13,0.47,0.30}{#1}}
\newcommand{\WarningTok}[1]{\textcolor[rgb]{0.37,0.37,0.37}{\textit{#1}}}

\providecommand{\tightlist}{%
  \setlength{\itemsep}{0pt}\setlength{\parskip}{0pt}}\usepackage{longtable,booktabs,array}
\usepackage{calc} % for calculating minipage widths
% Correct order of tables after \paragraph or \subparagraph
\usepackage{etoolbox}
\makeatletter
\patchcmd\longtable{\par}{\if@noskipsec\mbox{}\fi\par}{}{}
\makeatother
% Allow footnotes in longtable head/foot
\IfFileExists{footnotehyper.sty}{\usepackage{footnotehyper}}{\usepackage{footnote}}
\makesavenoteenv{longtable}
\usepackage{graphicx}
\makeatletter
\def\maxwidth{\ifdim\Gin@nat@width>\linewidth\linewidth\else\Gin@nat@width\fi}
\def\maxheight{\ifdim\Gin@nat@height>\textheight\textheight\else\Gin@nat@height\fi}
\makeatother
% Scale images if necessary, so that they will not overflow the page
% margins by default, and it is still possible to overwrite the defaults
% using explicit options in \includegraphics[width, height, ...]{}
\setkeys{Gin}{width=\maxwidth,height=\maxheight,keepaspectratio}
% Set default figure placement to htbp
\makeatletter
\def\fps@figure{htbp}
\makeatother

\usepackage{booktabs}
\usepackage{longtable}
\usepackage{array}
\usepackage{multirow}
\usepackage{wrapfig}
\usepackage{float}
\usepackage{colortbl}
\usepackage{pdflscape}
\usepackage{tabu}
\usepackage{threeparttable}
\usepackage{threeparttablex}
\usepackage[normalem]{ulem}
\usepackage{makecell}
\usepackage{xcolor}
\usepackage{fontspec}
\usepackage{multicol}
\usepackage{hhline}
\newlength\Oldarrayrulewidth
\newlength\Oldtabcolsep
\usepackage{hyperref}
\KOMAoption{captions}{tableheading}
\makeatletter
\@ifpackageloaded{caption}{}{\usepackage{caption}}
\AtBeginDocument{%
\ifdefined\contentsname
  \renewcommand*\contentsname{Table of contents}
\else
  \newcommand\contentsname{Table of contents}
\fi
\ifdefined\listfigurename
  \renewcommand*\listfigurename{List of Figures}
\else
  \newcommand\listfigurename{List of Figures}
\fi
\ifdefined\listtablename
  \renewcommand*\listtablename{List of Tables}
\else
  \newcommand\listtablename{List of Tables}
\fi
\ifdefined\figurename
  \renewcommand*\figurename{Figure}
\else
  \newcommand\figurename{Figure}
\fi
\ifdefined\tablename
  \renewcommand*\tablename{Table}
\else
  \newcommand\tablename{Table}
\fi
}
\@ifpackageloaded{float}{}{\usepackage{float}}
\floatstyle{ruled}
\@ifundefined{c@chapter}{\newfloat{codelisting}{h}{lop}}{\newfloat{codelisting}{h}{lop}[chapter]}
\floatname{codelisting}{Listing}
\newcommand*\listoflistings{\listof{codelisting}{List of Listings}}
\makeatother
\makeatletter
\makeatother
\makeatletter
\@ifpackageloaded{caption}{}{\usepackage{caption}}
\@ifpackageloaded{subcaption}{}{\usepackage{subcaption}}
\makeatother

\ifLuaTeX
  \usepackage{selnolig}  % disable illegal ligatures
\fi
\usepackage{bookmark}

\IfFileExists{xurl.sty}{\usepackage{xurl}}{} % add URL line breaks if available
\urlstyle{same} % disable monospaced font for URLs
\hypersetup{
  pdftitle={Tables walkthrough (Winter2025)},
  colorlinks=true,
  linkcolor={blue},
  filecolor={Maroon},
  citecolor={Blue},
  urlcolor={Blue},
  pdfcreator={LaTeX via pandoc}}


\title{Tables walkthrough (Winter2025)}
\author{}
\date{}

\begin{document}
\maketitle


Note: One important difference between R Markdown documents and Quarto
documents is that in Quarto chunk options are typically included in
special comments at the top of code chunks rather than within the line
that begins the chunk. For all potions, check out the
\href{https://quarto.org/docs/reference/cells/cells-knitr.html}{Quarto
documentation}.

\subsection{Tables: Basic ``Console
Style''}\label{tables-basic-console-style}

We can render tables in R documents in several different forms. To
start, you have your basic ``console style.'' This is more or less what
you'd see in the console you just ran the table (or called the object)
in an R script or directly in the console. See
Table~\ref{tbl-avgmpg-consolestyle}.

Note: For cross-reference of tables, you need to include a label in the
chunk options that starts with \texttt{tbl}. Then you can cite in-text
using the syntax ``@'' + ``tbl'' + ``label name''. Adding a label option
will enable the numbering of tables, too.

\begin{table}

\caption{\label{tbl-avgmpg-consolestyle}}

\centering{

\begin{Shaded}
\begin{Highlighting}[]
\NormalTok{mydata }\SpecialCharTok{\%\textgreater{}\%}
  \FunctionTok{group\_by}\NormalTok{(cyl) }\SpecialCharTok{\%\textgreater{}\%}
  \FunctionTok{summarize}\NormalTok{(}
    \AttributeTok{mean.mpg =} \FunctionTok{mean}\NormalTok{(mpg),}
    \AttributeTok{sd.mpg =} \FunctionTok{sd}\NormalTok{(mpg),}
    \AttributeTok{n.models =} \FunctionTok{n}\NormalTok{()}
\NormalTok{  )}
\end{Highlighting}
\end{Shaded}

\begin{verbatim}
# A tibble: 3 x 4
    cyl mean.mpg sd.mpg n.models
  <dbl>    <dbl>  <dbl>    <int>
1     4     26.7   4.51       11
2     6     19.7   1.45        7
3     8     15.1   2.56       14
\end{verbatim}

}

\end{table}%

\subsection{Tables: knitr and kable()}\label{tables-knitr-and-kable}

The \texttt{knitr} package contains the \texttt{kable()} function.
Passing a dataframe into this function will produce a formatted table
that's already pretty nice looking without needing additional
modification. You don't need to do anything more than pipe the df to
\texttt{kable()} to get this formatted table, but we can add optional
arguments like \texttt{caption}.

\begin{Shaded}
\begin{Highlighting}[]
\NormalTok{mydata }\SpecialCharTok{\%\textgreater{}\%}
  \FunctionTok{group\_by}\NormalTok{(cyl) }\SpecialCharTok{\%\textgreater{}\%}
  \FunctionTok{summarize}\NormalTok{(}
    \AttributeTok{mean.mpg =} \FunctionTok{mean}\NormalTok{(mpg),}
    \AttributeTok{sd.mpg =} \FunctionTok{sd}\NormalTok{(mpg),}
    \AttributeTok{n.models =} \FunctionTok{n}\NormalTok{()}
\NormalTok{  ) }\SpecialCharTok{\%\textgreater{}\%}
\NormalTok{  knitr}\SpecialCharTok{::}\FunctionTok{kable}\NormalTok{(}\AttributeTok{caption =} \StringTok{"mpg stats by number of cylinders"}\NormalTok{)}
\end{Highlighting}
\end{Shaded}

\begin{longtable}[]{@{}rrrr@{}}

\caption{\label{tbl-avgmpg-kablestyle}mpg stats by number of cylinders}

\tabularnewline

\toprule\noalign{}
cyl & mean.mpg & sd.mpg & n.models \\
\midrule\noalign{}
\endhead
\bottomrule\noalign{}
\endlastfoot
4 & 26.66364 & 4.509828 & 11 \\
6 & 19.74286 & 1.453567 & 7 \\
8 & 15.10000 & 2.560048 & 14 \\

\end{longtable}

When you run the code chunk in R Studio, it is clearly different from
the console version, but it's not exactly beautiful. However, that
output that appears beneath your code chunk isn't the same as what
you'll see when you actually render it. Some wrapper functions rely on
LaTeX formatting that can't render in R Studio. When you run the chunk
in your R notebook, you'll just see a blank box. The APA formatted table
will render in your Word or PDF document. Note that not all wrapper
functions will have this ``blank box problem.'' Other wrappers will not
only not show up in R Studio, but will only render in either Word
\emph{or} PDF format.

Back to \texttt{kable}, there is nearly unlimited customization you can
employ, especially if you use the
\href{https://haozhu233.github.io/kableExtra/}{\texttt{kableExtra}}
package (which is exactly what it sounds like). Here are a handful of
modifications I think you're most likely to want to know about:

\begin{enumerate}
\def\labelenumi{\arabic{enumi}.}
\tightlist
\item
  Add a caption
\item
  Change column names
\item
  Specify column alignment
\item
  Format columns
\item
  Style table size and position
\item
  Make row- and column-specific tweaks
\item
  Group rows, columns, and cells
\item
  Add table (foot)notes
\end{enumerate}

Many of the examples below are taken or adapted from the
\href{https://bookdown.org/yihui/rmarkdown-cookbook/tables.html}{R
Markdown Cookbook}.

These tables modify column names and alignment identically but with
slightly different syntax and add captions to describe each:

\begin{Shaded}
\begin{Highlighting}[]
\FunctionTok{kable}\NormalTok{(iris, }\CommentTok{\# iris is a built{-}in sample data in package knitr}
      \AttributeTok{col.names =} \FunctionTok{c}\NormalTok{(}\StringTok{"Sepal Length"}\NormalTok{, }\StringTok{"Sepal Width"}\NormalTok{,  }\StringTok{"Petal Length"}\NormalTok{, }\StringTok{"Petal Width"}\NormalTok{,  }\StringTok{"Species"}\NormalTok{), }
      \AttributeTok{align =} \FunctionTok{c}\NormalTok{(}\StringTok{"l"}\NormalTok{, }\StringTok{"r"}\NormalTok{, }\StringTok{"l"}\NormalTok{, }\StringTok{"r"}\NormalTok{, }\StringTok{"c"}\NormalTok{),}
      \AttributeTok{caption =} \StringTok{"Change each column name and text alignments, each with a list of strings."}\NormalTok{)}
\end{Highlighting}
\end{Shaded}

\begin{longtable}[]{@{}lrlrc@{}}

\caption{\label{tbl-column-names-alignment1}Change each column name and
text alignments, each with a list of strings.}

\tabularnewline

\toprule\noalign{}
Sepal Length & Sepal Width & Petal Length & Petal Width & Species \\
\midrule\noalign{}
\endhead
\bottomrule\noalign{}
\endlastfoot
5.1 & 3.5 & 1.4 & 0.2 & setosa \\
4.9 & 3.0 & 1.4 & 0.2 & setosa \\
4.7 & 3.2 & 1.3 & 0.2 & setosa \\
4.6 & 3.1 & 1.5 & 0.2 & setosa \\
5.0 & 3.6 & 1.4 & 0.2 & setosa \\
5.4 & 3.9 & 1.7 & 0.4 & setosa \\
4.6 & 3.4 & 1.4 & 0.3 & setosa \\
5.0 & 3.4 & 1.5 & 0.2 & setosa \\
4.4 & 2.9 & 1.4 & 0.2 & setosa \\
4.9 & 3.1 & 1.5 & 0.1 & setosa \\
5.4 & 3.7 & 1.5 & 0.2 & setosa \\
4.8 & 3.4 & 1.6 & 0.2 & setosa \\
4.8 & 3.0 & 1.4 & 0.1 & setosa \\
4.3 & 3.0 & 1.1 & 0.1 & setosa \\
5.8 & 4.0 & 1.2 & 0.2 & setosa \\
5.7 & 4.4 & 1.5 & 0.4 & setosa \\
5.4 & 3.9 & 1.3 & 0.4 & setosa \\
5.1 & 3.5 & 1.4 & 0.3 & setosa \\
5.7 & 3.8 & 1.7 & 0.3 & setosa \\
5.1 & 3.8 & 1.5 & 0.3 & setosa \\
5.4 & 3.4 & 1.7 & 0.2 & setosa \\
5.1 & 3.7 & 1.5 & 0.4 & setosa \\
4.6 & 3.6 & 1.0 & 0.2 & setosa \\
5.1 & 3.3 & 1.7 & 0.5 & setosa \\
4.8 & 3.4 & 1.9 & 0.2 & setosa \\
5.0 & 3.0 & 1.6 & 0.2 & setosa \\
5.0 & 3.4 & 1.6 & 0.4 & setosa \\
5.2 & 3.5 & 1.5 & 0.2 & setosa \\
5.2 & 3.4 & 1.4 & 0.2 & setosa \\
4.7 & 3.2 & 1.6 & 0.2 & setosa \\
4.8 & 3.1 & 1.6 & 0.2 & setosa \\
5.4 & 3.4 & 1.5 & 0.4 & setosa \\
5.2 & 4.1 & 1.5 & 0.1 & setosa \\
5.5 & 4.2 & 1.4 & 0.2 & setosa \\
4.9 & 3.1 & 1.5 & 0.2 & setosa \\
5.0 & 3.2 & 1.2 & 0.2 & setosa \\
5.5 & 3.5 & 1.3 & 0.2 & setosa \\
4.9 & 3.6 & 1.4 & 0.1 & setosa \\
4.4 & 3.0 & 1.3 & 0.2 & setosa \\
5.1 & 3.4 & 1.5 & 0.2 & setosa \\
5.0 & 3.5 & 1.3 & 0.3 & setosa \\
4.5 & 2.3 & 1.3 & 0.3 & setosa \\
4.4 & 3.2 & 1.3 & 0.2 & setosa \\
5.0 & 3.5 & 1.6 & 0.6 & setosa \\
5.1 & 3.8 & 1.9 & 0.4 & setosa \\
4.8 & 3.0 & 1.4 & 0.3 & setosa \\
5.1 & 3.8 & 1.6 & 0.2 & setosa \\
4.6 & 3.2 & 1.4 & 0.2 & setosa \\
5.3 & 3.7 & 1.5 & 0.2 & setosa \\
5.0 & 3.3 & 1.4 & 0.2 & setosa \\
7.0 & 3.2 & 4.7 & 1.4 & versicolor \\
6.4 & 3.2 & 4.5 & 1.5 & versicolor \\
6.9 & 3.1 & 4.9 & 1.5 & versicolor \\
5.5 & 2.3 & 4.0 & 1.3 & versicolor \\
6.5 & 2.8 & 4.6 & 1.5 & versicolor \\
5.7 & 2.8 & 4.5 & 1.3 & versicolor \\
6.3 & 3.3 & 4.7 & 1.6 & versicolor \\
4.9 & 2.4 & 3.3 & 1.0 & versicolor \\
6.6 & 2.9 & 4.6 & 1.3 & versicolor \\
5.2 & 2.7 & 3.9 & 1.4 & versicolor \\
5.0 & 2.0 & 3.5 & 1.0 & versicolor \\
5.9 & 3.0 & 4.2 & 1.5 & versicolor \\
6.0 & 2.2 & 4.0 & 1.0 & versicolor \\
6.1 & 2.9 & 4.7 & 1.4 & versicolor \\
5.6 & 2.9 & 3.6 & 1.3 & versicolor \\
6.7 & 3.1 & 4.4 & 1.4 & versicolor \\
5.6 & 3.0 & 4.5 & 1.5 & versicolor \\
5.8 & 2.7 & 4.1 & 1.0 & versicolor \\
6.2 & 2.2 & 4.5 & 1.5 & versicolor \\
5.6 & 2.5 & 3.9 & 1.1 & versicolor \\
5.9 & 3.2 & 4.8 & 1.8 & versicolor \\
6.1 & 2.8 & 4.0 & 1.3 & versicolor \\
6.3 & 2.5 & 4.9 & 1.5 & versicolor \\
6.1 & 2.8 & 4.7 & 1.2 & versicolor \\
6.4 & 2.9 & 4.3 & 1.3 & versicolor \\
6.6 & 3.0 & 4.4 & 1.4 & versicolor \\
6.8 & 2.8 & 4.8 & 1.4 & versicolor \\
6.7 & 3.0 & 5.0 & 1.7 & versicolor \\
6.0 & 2.9 & 4.5 & 1.5 & versicolor \\
5.7 & 2.6 & 3.5 & 1.0 & versicolor \\
5.5 & 2.4 & 3.8 & 1.1 & versicolor \\
5.5 & 2.4 & 3.7 & 1.0 & versicolor \\
5.8 & 2.7 & 3.9 & 1.2 & versicolor \\
6.0 & 2.7 & 5.1 & 1.6 & versicolor \\
5.4 & 3.0 & 4.5 & 1.5 & versicolor \\
6.0 & 3.4 & 4.5 & 1.6 & versicolor \\
6.7 & 3.1 & 4.7 & 1.5 & versicolor \\
6.3 & 2.3 & 4.4 & 1.3 & versicolor \\
5.6 & 3.0 & 4.1 & 1.3 & versicolor \\
5.5 & 2.5 & 4.0 & 1.3 & versicolor \\
5.5 & 2.6 & 4.4 & 1.2 & versicolor \\
6.1 & 3.0 & 4.6 & 1.4 & versicolor \\
5.8 & 2.6 & 4.0 & 1.2 & versicolor \\
5.0 & 2.3 & 3.3 & 1.0 & versicolor \\
5.6 & 2.7 & 4.2 & 1.3 & versicolor \\
5.7 & 3.0 & 4.2 & 1.2 & versicolor \\
5.7 & 2.9 & 4.2 & 1.3 & versicolor \\
6.2 & 2.9 & 4.3 & 1.3 & versicolor \\
5.1 & 2.5 & 3.0 & 1.1 & versicolor \\
5.7 & 2.8 & 4.1 & 1.3 & versicolor \\
6.3 & 3.3 & 6.0 & 2.5 & virginica \\
5.8 & 2.7 & 5.1 & 1.9 & virginica \\
7.1 & 3.0 & 5.9 & 2.1 & virginica \\
6.3 & 2.9 & 5.6 & 1.8 & virginica \\
6.5 & 3.0 & 5.8 & 2.2 & virginica \\
7.6 & 3.0 & 6.6 & 2.1 & virginica \\
4.9 & 2.5 & 4.5 & 1.7 & virginica \\
7.3 & 2.9 & 6.3 & 1.8 & virginica \\
6.7 & 2.5 & 5.8 & 1.8 & virginica \\
7.2 & 3.6 & 6.1 & 2.5 & virginica \\
6.5 & 3.2 & 5.1 & 2.0 & virginica \\
6.4 & 2.7 & 5.3 & 1.9 & virginica \\
6.8 & 3.0 & 5.5 & 2.1 & virginica \\
5.7 & 2.5 & 5.0 & 2.0 & virginica \\
5.8 & 2.8 & 5.1 & 2.4 & virginica \\
6.4 & 3.2 & 5.3 & 2.3 & virginica \\
6.5 & 3.0 & 5.5 & 1.8 & virginica \\
7.7 & 3.8 & 6.7 & 2.2 & virginica \\
7.7 & 2.6 & 6.9 & 2.3 & virginica \\
6.0 & 2.2 & 5.0 & 1.5 & virginica \\
6.9 & 3.2 & 5.7 & 2.3 & virginica \\
5.6 & 2.8 & 4.9 & 2.0 & virginica \\
7.7 & 2.8 & 6.7 & 2.0 & virginica \\
6.3 & 2.7 & 4.9 & 1.8 & virginica \\
6.7 & 3.3 & 5.7 & 2.1 & virginica \\
7.2 & 3.2 & 6.0 & 1.8 & virginica \\
6.2 & 2.8 & 4.8 & 1.8 & virginica \\
6.1 & 3.0 & 4.9 & 1.8 & virginica \\
6.4 & 2.8 & 5.6 & 2.1 & virginica \\
7.2 & 3.0 & 5.8 & 1.6 & virginica \\
7.4 & 2.8 & 6.1 & 1.9 & virginica \\
7.9 & 3.8 & 6.4 & 2.0 & virginica \\
6.4 & 2.8 & 5.6 & 2.2 & virginica \\
6.3 & 2.8 & 5.1 & 1.5 & virginica \\
6.1 & 2.6 & 5.6 & 1.4 & virginica \\
7.7 & 3.0 & 6.1 & 2.3 & virginica \\
6.3 & 3.4 & 5.6 & 2.4 & virginica \\
6.4 & 3.1 & 5.5 & 1.8 & virginica \\
6.0 & 3.0 & 4.8 & 1.8 & virginica \\
6.9 & 3.1 & 5.4 & 2.1 & virginica \\
6.7 & 3.1 & 5.6 & 2.4 & virginica \\
6.9 & 3.1 & 5.1 & 2.3 & virginica \\
5.8 & 2.7 & 5.1 & 1.9 & virginica \\
6.8 & 3.2 & 5.9 & 2.3 & virginica \\
6.7 & 3.3 & 5.7 & 2.5 & virginica \\
6.7 & 3.0 & 5.2 & 2.3 & virginica \\
6.3 & 2.5 & 5.0 & 1.9 & virginica \\
6.5 & 3.0 & 5.2 & 2.0 & virginica \\
6.2 & 3.4 & 5.4 & 2.3 & virginica \\
5.9 & 3.0 & 5.1 & 1.8 & virginica \\

\end{longtable}

\begin{Shaded}
\begin{Highlighting}[]
\FunctionTok{kable}\NormalTok{(iris, }
      \AttributeTok{col.names =} \FunctionTok{gsub}\NormalTok{(}\StringTok{"[.]"}\NormalTok{, }\StringTok{" "}\NormalTok{, }\FunctionTok{names}\NormalTok{(iris)), }
      \AttributeTok{align =} \StringTok{"lrlrc"}\NormalTok{,}
      \AttributeTok{caption =} \StringTok{"Use \textquotesingle{}gsub\textquotesingle{} function to replace periods with spaces and change text alignments with a single \textquotesingle{}shortcut\textquotesingle{} string."}\NormalTok{)}
\end{Highlighting}
\end{Shaded}

\begin{longtable}[]{@{}lrlrc@{}}

\caption{\label{tbl-column-names-alignment2}Use `gsub' function to
replace periods with spaces and change text alignments with a single
`shortcut' string.}

\tabularnewline

\toprule\noalign{}
Sepal Length & Sepal Width & Petal Length & Petal Width & Species \\
\midrule\noalign{}
\endhead
\bottomrule\noalign{}
\endlastfoot
5.1 & 3.5 & 1.4 & 0.2 & setosa \\
4.9 & 3.0 & 1.4 & 0.2 & setosa \\
4.7 & 3.2 & 1.3 & 0.2 & setosa \\
4.6 & 3.1 & 1.5 & 0.2 & setosa \\
5.0 & 3.6 & 1.4 & 0.2 & setosa \\
5.4 & 3.9 & 1.7 & 0.4 & setosa \\
4.6 & 3.4 & 1.4 & 0.3 & setosa \\
5.0 & 3.4 & 1.5 & 0.2 & setosa \\
4.4 & 2.9 & 1.4 & 0.2 & setosa \\
4.9 & 3.1 & 1.5 & 0.1 & setosa \\
5.4 & 3.7 & 1.5 & 0.2 & setosa \\
4.8 & 3.4 & 1.6 & 0.2 & setosa \\
4.8 & 3.0 & 1.4 & 0.1 & setosa \\
4.3 & 3.0 & 1.1 & 0.1 & setosa \\
5.8 & 4.0 & 1.2 & 0.2 & setosa \\
5.7 & 4.4 & 1.5 & 0.4 & setosa \\
5.4 & 3.9 & 1.3 & 0.4 & setosa \\
5.1 & 3.5 & 1.4 & 0.3 & setosa \\
5.7 & 3.8 & 1.7 & 0.3 & setosa \\
5.1 & 3.8 & 1.5 & 0.3 & setosa \\
5.4 & 3.4 & 1.7 & 0.2 & setosa \\
5.1 & 3.7 & 1.5 & 0.4 & setosa \\
4.6 & 3.6 & 1.0 & 0.2 & setosa \\
5.1 & 3.3 & 1.7 & 0.5 & setosa \\
4.8 & 3.4 & 1.9 & 0.2 & setosa \\
5.0 & 3.0 & 1.6 & 0.2 & setosa \\
5.0 & 3.4 & 1.6 & 0.4 & setosa \\
5.2 & 3.5 & 1.5 & 0.2 & setosa \\
5.2 & 3.4 & 1.4 & 0.2 & setosa \\
4.7 & 3.2 & 1.6 & 0.2 & setosa \\
4.8 & 3.1 & 1.6 & 0.2 & setosa \\
5.4 & 3.4 & 1.5 & 0.4 & setosa \\
5.2 & 4.1 & 1.5 & 0.1 & setosa \\
5.5 & 4.2 & 1.4 & 0.2 & setosa \\
4.9 & 3.1 & 1.5 & 0.2 & setosa \\
5.0 & 3.2 & 1.2 & 0.2 & setosa \\
5.5 & 3.5 & 1.3 & 0.2 & setosa \\
4.9 & 3.6 & 1.4 & 0.1 & setosa \\
4.4 & 3.0 & 1.3 & 0.2 & setosa \\
5.1 & 3.4 & 1.5 & 0.2 & setosa \\
5.0 & 3.5 & 1.3 & 0.3 & setosa \\
4.5 & 2.3 & 1.3 & 0.3 & setosa \\
4.4 & 3.2 & 1.3 & 0.2 & setosa \\
5.0 & 3.5 & 1.6 & 0.6 & setosa \\
5.1 & 3.8 & 1.9 & 0.4 & setosa \\
4.8 & 3.0 & 1.4 & 0.3 & setosa \\
5.1 & 3.8 & 1.6 & 0.2 & setosa \\
4.6 & 3.2 & 1.4 & 0.2 & setosa \\
5.3 & 3.7 & 1.5 & 0.2 & setosa \\
5.0 & 3.3 & 1.4 & 0.2 & setosa \\
7.0 & 3.2 & 4.7 & 1.4 & versicolor \\
6.4 & 3.2 & 4.5 & 1.5 & versicolor \\
6.9 & 3.1 & 4.9 & 1.5 & versicolor \\
5.5 & 2.3 & 4.0 & 1.3 & versicolor \\
6.5 & 2.8 & 4.6 & 1.5 & versicolor \\
5.7 & 2.8 & 4.5 & 1.3 & versicolor \\
6.3 & 3.3 & 4.7 & 1.6 & versicolor \\
4.9 & 2.4 & 3.3 & 1.0 & versicolor \\
6.6 & 2.9 & 4.6 & 1.3 & versicolor \\
5.2 & 2.7 & 3.9 & 1.4 & versicolor \\
5.0 & 2.0 & 3.5 & 1.0 & versicolor \\
5.9 & 3.0 & 4.2 & 1.5 & versicolor \\
6.0 & 2.2 & 4.0 & 1.0 & versicolor \\
6.1 & 2.9 & 4.7 & 1.4 & versicolor \\
5.6 & 2.9 & 3.6 & 1.3 & versicolor \\
6.7 & 3.1 & 4.4 & 1.4 & versicolor \\
5.6 & 3.0 & 4.5 & 1.5 & versicolor \\
5.8 & 2.7 & 4.1 & 1.0 & versicolor \\
6.2 & 2.2 & 4.5 & 1.5 & versicolor \\
5.6 & 2.5 & 3.9 & 1.1 & versicolor \\
5.9 & 3.2 & 4.8 & 1.8 & versicolor \\
6.1 & 2.8 & 4.0 & 1.3 & versicolor \\
6.3 & 2.5 & 4.9 & 1.5 & versicolor \\
6.1 & 2.8 & 4.7 & 1.2 & versicolor \\
6.4 & 2.9 & 4.3 & 1.3 & versicolor \\
6.6 & 3.0 & 4.4 & 1.4 & versicolor \\
6.8 & 2.8 & 4.8 & 1.4 & versicolor \\
6.7 & 3.0 & 5.0 & 1.7 & versicolor \\
6.0 & 2.9 & 4.5 & 1.5 & versicolor \\
5.7 & 2.6 & 3.5 & 1.0 & versicolor \\
5.5 & 2.4 & 3.8 & 1.1 & versicolor \\
5.5 & 2.4 & 3.7 & 1.0 & versicolor \\
5.8 & 2.7 & 3.9 & 1.2 & versicolor \\
6.0 & 2.7 & 5.1 & 1.6 & versicolor \\
5.4 & 3.0 & 4.5 & 1.5 & versicolor \\
6.0 & 3.4 & 4.5 & 1.6 & versicolor \\
6.7 & 3.1 & 4.7 & 1.5 & versicolor \\
6.3 & 2.3 & 4.4 & 1.3 & versicolor \\
5.6 & 3.0 & 4.1 & 1.3 & versicolor \\
5.5 & 2.5 & 4.0 & 1.3 & versicolor \\
5.5 & 2.6 & 4.4 & 1.2 & versicolor \\
6.1 & 3.0 & 4.6 & 1.4 & versicolor \\
5.8 & 2.6 & 4.0 & 1.2 & versicolor \\
5.0 & 2.3 & 3.3 & 1.0 & versicolor \\
5.6 & 2.7 & 4.2 & 1.3 & versicolor \\
5.7 & 3.0 & 4.2 & 1.2 & versicolor \\
5.7 & 2.9 & 4.2 & 1.3 & versicolor \\
6.2 & 2.9 & 4.3 & 1.3 & versicolor \\
5.1 & 2.5 & 3.0 & 1.1 & versicolor \\
5.7 & 2.8 & 4.1 & 1.3 & versicolor \\
6.3 & 3.3 & 6.0 & 2.5 & virginica \\
5.8 & 2.7 & 5.1 & 1.9 & virginica \\
7.1 & 3.0 & 5.9 & 2.1 & virginica \\
6.3 & 2.9 & 5.6 & 1.8 & virginica \\
6.5 & 3.0 & 5.8 & 2.2 & virginica \\
7.6 & 3.0 & 6.6 & 2.1 & virginica \\
4.9 & 2.5 & 4.5 & 1.7 & virginica \\
7.3 & 2.9 & 6.3 & 1.8 & virginica \\
6.7 & 2.5 & 5.8 & 1.8 & virginica \\
7.2 & 3.6 & 6.1 & 2.5 & virginica \\
6.5 & 3.2 & 5.1 & 2.0 & virginica \\
6.4 & 2.7 & 5.3 & 1.9 & virginica \\
6.8 & 3.0 & 5.5 & 2.1 & virginica \\
5.7 & 2.5 & 5.0 & 2.0 & virginica \\
5.8 & 2.8 & 5.1 & 2.4 & virginica \\
6.4 & 3.2 & 5.3 & 2.3 & virginica \\
6.5 & 3.0 & 5.5 & 1.8 & virginica \\
7.7 & 3.8 & 6.7 & 2.2 & virginica \\
7.7 & 2.6 & 6.9 & 2.3 & virginica \\
6.0 & 2.2 & 5.0 & 1.5 & virginica \\
6.9 & 3.2 & 5.7 & 2.3 & virginica \\
5.6 & 2.8 & 4.9 & 2.0 & virginica \\
7.7 & 2.8 & 6.7 & 2.0 & virginica \\
6.3 & 2.7 & 4.9 & 1.8 & virginica \\
6.7 & 3.3 & 5.7 & 2.1 & virginica \\
7.2 & 3.2 & 6.0 & 1.8 & virginica \\
6.2 & 2.8 & 4.8 & 1.8 & virginica \\
6.1 & 3.0 & 4.9 & 1.8 & virginica \\
6.4 & 2.8 & 5.6 & 2.1 & virginica \\
7.2 & 3.0 & 5.8 & 1.6 & virginica \\
7.4 & 2.8 & 6.1 & 1.9 & virginica \\
7.9 & 3.8 & 6.4 & 2.0 & virginica \\
6.4 & 2.8 & 5.6 & 2.2 & virginica \\
6.3 & 2.8 & 5.1 & 1.5 & virginica \\
6.1 & 2.6 & 5.6 & 1.4 & virginica \\
7.7 & 3.0 & 6.1 & 2.3 & virginica \\
6.3 & 3.4 & 5.6 & 2.4 & virginica \\
6.4 & 3.1 & 5.5 & 1.8 & virginica \\
6.0 & 3.0 & 4.8 & 1.8 & virginica \\
6.9 & 3.1 & 5.4 & 2.1 & virginica \\
6.7 & 3.1 & 5.6 & 2.4 & virginica \\
6.9 & 3.1 & 5.1 & 2.3 & virginica \\
5.8 & 2.7 & 5.1 & 1.9 & virginica \\
6.8 & 3.2 & 5.9 & 2.3 & virginica \\
6.7 & 3.3 & 5.7 & 2.5 & virginica \\
6.7 & 3.0 & 5.2 & 2.3 & virginica \\
6.3 & 2.5 & 5.0 & 1.9 & virginica \\
6.5 & 3.0 & 5.2 & 2.0 & virginica \\
6.2 & 3.4 & 5.4 & 2.3 & virginica \\
5.9 & 3.0 & 5.1 & 1.8 & virginica \\

\end{longtable}

These tables specify format of numeric columns:

\begin{Shaded}
\begin{Highlighting}[]
\NormalTok{d }\OtherTok{\textless{}{-}} \FunctionTok{cbind}\NormalTok{(}
  \AttributeTok{X1 =} \FunctionTok{runif}\NormalTok{(}\DecValTok{3}\NormalTok{), }
  \AttributeTok{X2 =} \DecValTok{10}\SpecialCharTok{\^{}}\FunctionTok{c}\NormalTok{(}\DecValTok{3}\NormalTok{, }\DecValTok{5}\NormalTok{, }\DecValTok{7}\NormalTok{), }
  \AttributeTok{X3 =} \FunctionTok{rnorm}\NormalTok{(}\DecValTok{3}\NormalTok{, }\DecValTok{0}\NormalTok{, }\DecValTok{1000}\NormalTok{))}

\FunctionTok{kable}\NormalTok{(d, }\AttributeTok{digits =} \DecValTok{4}\NormalTok{,}
      \AttributeTok{caption =} \StringTok{"All numeric data in all columns display at most 4 decimal places"}\NormalTok{)}
\end{Highlighting}
\end{Shaded}

\begin{longtable}[]{@{}rrr@{}}

\caption{\label{tbl-column-formats1}All numeric data in all columns
display at most 4 decimal places}

\tabularnewline

\toprule\noalign{}
X1 & X2 & X3 \\
\midrule\noalign{}
\endhead
\bottomrule\noalign{}
\endlastfoot
0.9148 & 1e+03 & 955.9356 \\
0.9371 & 1e+05 & 47.8847 \\
0.2861 & 1e+07 & -1104.5994 \\

\end{longtable}

\begin{Shaded}
\begin{Highlighting}[]
\FunctionTok{kable}\NormalTok{(d, }\AttributeTok{digits =} \FunctionTok{c}\NormalTok{(}\DecValTok{5}\NormalTok{, }\DecValTok{0}\NormalTok{, }\DecValTok{2}\NormalTok{),}
      \AttributeTok{caption =} \StringTok{"Round columns to 5, 0, and 2 digits (respectively)."}\NormalTok{)}
\end{Highlighting}
\end{Shaded}

\begin{longtable}[]{@{}rrr@{}}

\caption{\label{tbl-column-formats2}Round columns to 5, 0, and 2 digits
(respectively).}

\tabularnewline

\toprule\noalign{}
X1 & X2 & X3 \\
\midrule\noalign{}
\endhead
\bottomrule\noalign{}
\endlastfoot
0.91481 & 1e+03 & 955.94 \\
0.93708 & 1e+05 & 47.88 \\
0.28614 & 1e+07 & -1104.60 \\

\end{longtable}

\begin{Shaded}
\begin{Highlighting}[]
\FunctionTok{kable}\NormalTok{(d, }\AttributeTok{digits =} \DecValTok{3}\NormalTok{, }
      \AttributeTok{format.args =} \FunctionTok{list}\NormalTok{(}\AttributeTok{big.mark =} \StringTok{","}\NormalTok{, }\CommentTok{\# Use US notation of including a comma (vs period) every three digits}
                         \AttributeTok{scientific =} \ConstantTok{FALSE}\NormalTok{),}
      \AttributeTok{caption =} \StringTok{"Round all data to max 3 decimal places and do not use scientific notaiton."}\NormalTok{)}
\end{Highlighting}
\end{Shaded}

\begin{longtable}[]{@{}rrr@{}}

\caption{\label{tbl-column-formats3}Round all data to max 3 decimal
places and do not use scientific notaiton.}

\tabularnewline

\toprule\noalign{}
X1 & X2 & X3 \\
\midrule\noalign{}
\endhead
\bottomrule\noalign{}
\endlastfoot
0.915 & 1,000 & 955.936 \\
0.937 & 100,000 & 47.885 \\
0.286 & 10,000,000 & -1,104.599 \\

\end{longtable}

For the second half of the list (5-8), we need the \texttt{kableExtra}
package. Most things that this package can do will work in both Word and
PDF outputs, but Word does not support LaTeX formatting. As a result,
the results are pretty iffy. Sticking with PDF output format is
recommended. If you need to render to a Word doc, refer to the package
documentation for suggestions on how to do so.

This package includes the \texttt{kbl()} function, which is
\emph{identical} to \texttt{kable()}. If you have \texttt{kableExtra}
loaded you can use either, but the advantage of using \texttt{kbl()} is
that it will throw an obvious error if you don't have the package
loaded, which you might not notice otherwise.

To use styling functions in the \texttt{kableExtra} package, you'll
create a kable with \texttt{kable()} or \texttt{kbl()} and pipe
(\texttt{\%\textgreater{}\%}) it into the style function. Examples in
this section come from the \texttt{kableExtra} package
\href{https://haozhu233.github.io/kableExtra/}{documentation}.

\begin{Shaded}
\begin{Highlighting}[]
\FunctionTok{library}\NormalTok{(kableExtra)}
\end{Highlighting}
\end{Shaded}

\begin{verbatim}

Attaching package: 'kableExtra'
\end{verbatim}

\begin{verbatim}
The following object is masked from 'package:dplyr':

    group_rows
\end{verbatim}

\begin{Shaded}
\begin{Highlighting}[]
\NormalTok{dt }\OtherTok{\textless{}{-}}\NormalTok{ mtcars[}\DecValTok{1}\SpecialCharTok{:}\DecValTok{5}\NormalTok{, }\DecValTok{1}\SpecialCharTok{:}\DecValTok{6}\NormalTok{]}
\end{Highlighting}
\end{Shaded}

By default, knitr will place tables wherever it thinks the best place in
your document for that table will be. This is often great, but sometimes
it's a very bad guesser. It will also guess the size/scale your table
should be. You can control these things yourself.

In Table~\ref{tbl-hold-position}, the \texttt{hold\_position} style
option ``pins'' the table where you call it. Most examples from here on
will include this argument.

\begin{Shaded}
\begin{Highlighting}[]
\FunctionTok{kbl}\NormalTok{(dt, }
    \AttributeTok{caption =} \StringTok{"Table will appear exactly where you call it."}
\NormalTok{    ) }\SpecialCharTok{\%\textgreater{}\%}
  \FunctionTok{kable\_styling}\NormalTok{(}\AttributeTok{latex\_options =} \StringTok{"hold\_position"}\NormalTok{)}
\end{Highlighting}
\end{Shaded}

\begin{table}

\caption{\label{tbl-hold-position}}

\centering{

[!h]
\centering
\caption{Table will appear exactly where you call it.}
\centering
\begin{tabular}[t]{l|r|r|r|r|r|r}
\hline
  & mpg & cyl & disp & hp & drat & wt\\
\hline
Mazda RX4 & 21.0 & 6 & 160 & 110 & 3.90 & 2.620\\
\hline
Mazda RX4 Wag & 21.0 & 6 & 160 & 110 & 3.90 & 2.875\\
\hline
Datsun 710 & 22.8 & 4 & 108 & 93 & 3.85 & 2.320\\
\hline
Hornet 4 Drive & 21.4 & 6 & 258 & 110 & 3.08 & 3.215\\
\hline
Hornet Sportabout & 18.7 & 8 & 360 & 175 & 3.15 & 3.440\\
\hline
\end{tabular}

}

\end{table}%

In Table~\ref{tbl-size-scaledown}, it shows the use of the
\texttt{scale\_down} option (with \texttt{hold\_position}) to scale a
wide table down to fit on the page. This can be used instead of or in
addition to printing the table in landscape page format. See the
\texttt{kableExtra} docs for how to do that.

\begin{Shaded}
\begin{Highlighting}[]
\FunctionTok{kbl}\NormalTok{(}\FunctionTok{cbind}\NormalTok{(dt, dt, dt), }\CommentTok{\# create a very wide table}
    \AttributeTok{caption =} \StringTok{"When you have a wide table that will normally go out of the page, and you want to scale down the table to fit the page, you can use the scale down option."}\NormalTok{) }\SpecialCharTok{\%\textgreater{}\%}
  \FunctionTok{kable\_styling}\NormalTok{(}\AttributeTok{latex\_options =} \FunctionTok{c}\NormalTok{(}\StringTok{"striped"}\NormalTok{, }\StringTok{"scale\_down"}\NormalTok{, }\StringTok{"hold\_position"}\NormalTok{))}
\end{Highlighting}
\end{Shaded}

\begin{table}

\caption{\label{tbl-size-scaledown}}

\centering{

[!h]
\centering
\caption{When you have a wide table that will normally go out of the page, and you want to scale down the table to fit the page, you can use the scale down option.}
\centering
\resizebox{\ifdim\width>\linewidth\linewidth\else\width\fi}{!}{
\begin{tabular}[t]{l|r|r|r|r|r|r|r|r|r|r|r|r|r|r|r|r|r|r}
\hline
  & mpg & cyl & disp & hp & drat & wt & mpg & cyl & disp & hp & drat & wt & mpg & cyl & disp & hp & drat & wt\\
\hline
\cellcolor{gray!10}{Mazda RX4} & \cellcolor{gray!10}{21.0} & \cellcolor{gray!10}{6} & \cellcolor{gray!10}{160} & \cellcolor{gray!10}{110} & \cellcolor{gray!10}{3.90} & \cellcolor{gray!10}{2.620} & \cellcolor{gray!10}{21.0} & \cellcolor{gray!10}{6} & \cellcolor{gray!10}{160} & \cellcolor{gray!10}{110} & \cellcolor{gray!10}{3.90} & \cellcolor{gray!10}{2.620} & \cellcolor{gray!10}{21.0} & \cellcolor{gray!10}{6} & \cellcolor{gray!10}{160} & \cellcolor{gray!10}{110} & \cellcolor{gray!10}{3.90} & \cellcolor{gray!10}{2.620}\\
\hline
Mazda RX4 Wag & 21.0 & 6 & 160 & 110 & 3.90 & 2.875 & 21.0 & 6 & 160 & 110 & 3.90 & 2.875 & 21.0 & 6 & 160 & 110 & 3.90 & 2.875\\
\hline
\cellcolor{gray!10}{Datsun 710} & \cellcolor{gray!10}{22.8} & \cellcolor{gray!10}{4} & \cellcolor{gray!10}{108} & \cellcolor{gray!10}{93} & \cellcolor{gray!10}{3.85} & \cellcolor{gray!10}{2.320} & \cellcolor{gray!10}{22.8} & \cellcolor{gray!10}{4} & \cellcolor{gray!10}{108} & \cellcolor{gray!10}{93} & \cellcolor{gray!10}{3.85} & \cellcolor{gray!10}{2.320} & \cellcolor{gray!10}{22.8} & \cellcolor{gray!10}{4} & \cellcolor{gray!10}{108} & \cellcolor{gray!10}{93} & \cellcolor{gray!10}{3.85} & \cellcolor{gray!10}{2.320}\\
\hline
Hornet 4 Drive & 21.4 & 6 & 258 & 110 & 3.08 & 3.215 & 21.4 & 6 & 258 & 110 & 3.08 & 3.215 & 21.4 & 6 & 258 & 110 & 3.08 & 3.215\\
\hline
\cellcolor{gray!10}{Hornet Sportabout} & \cellcolor{gray!10}{18.7} & \cellcolor{gray!10}{8} & \cellcolor{gray!10}{360} & \cellcolor{gray!10}{175} & \cellcolor{gray!10}{3.15} & \cellcolor{gray!10}{3.440} & \cellcolor{gray!10}{18.7} & \cellcolor{gray!10}{8} & \cellcolor{gray!10}{360} & \cellcolor{gray!10}{175} & \cellcolor{gray!10}{3.15} & \cellcolor{gray!10}{3.440} & \cellcolor{gray!10}{18.7} & \cellcolor{gray!10}{8} & \cellcolor{gray!10}{360} & \cellcolor{gray!10}{175} & \cellcolor{gray!10}{3.15} & \cellcolor{gray!10}{3.440}\\
\hline
\end{tabular}}

}

\end{table}%

You can also scale \emph{up} a table to make a small/narrow table full
width with either the \texttt{scale\_up} option (see
Table~\ref{tbl-size-scaleup} ) or \texttt{full\_width} argument (see
Table~\ref{tbl-size-fullwidth} ).

\begin{Shaded}
\begin{Highlighting}[]
\FunctionTok{kbl}\NormalTok{(}\FunctionTok{cbind}\NormalTok{(dt),}
    \AttributeTok{caption =} \StringTok{"Use the scale up option to stretch to full page width."}\NormalTok{) }\SpecialCharTok{\%\textgreater{}\%}
  \FunctionTok{kable\_styling}\NormalTok{(}\AttributeTok{latex\_options =} \FunctionTok{c}\NormalTok{(}\StringTok{"striped"}\NormalTok{, }\StringTok{"scale\_up"}\NormalTok{, }\StringTok{"hold\_position"}\NormalTok{))}
\end{Highlighting}
\end{Shaded}

\begin{table}

\caption{\label{tbl-size-scaleup}}

\centering{

[!h]
\centering
\caption{Use the scale up option to stretch to full page width.}
\centering
\resizebox{\ifdim\width<\linewidth\linewidth\else\width\fi}{!}{
\begin{tabular}[t]{l|r|r|r|r|r|r}
\hline
  & mpg & cyl & disp & hp & drat & wt\\
\hline
\cellcolor{gray!10}{Mazda RX4} & \cellcolor{gray!10}{21.0} & \cellcolor{gray!10}{6} & \cellcolor{gray!10}{160} & \cellcolor{gray!10}{110} & \cellcolor{gray!10}{3.90} & \cellcolor{gray!10}{2.620}\\
\hline
Mazda RX4 Wag & 21.0 & 6 & 160 & 110 & 3.90 & 2.875\\
\hline
\cellcolor{gray!10}{Datsun 710} & \cellcolor{gray!10}{22.8} & \cellcolor{gray!10}{4} & \cellcolor{gray!10}{108} & \cellcolor{gray!10}{93} & \cellcolor{gray!10}{3.85} & \cellcolor{gray!10}{2.320}\\
\hline
Hornet 4 Drive & 21.4 & 6 & 258 & 110 & 3.08 & 3.215\\
\hline
\cellcolor{gray!10}{Hornet Sportabout} & \cellcolor{gray!10}{18.7} & \cellcolor{gray!10}{8} & \cellcolor{gray!10}{360} & \cellcolor{gray!10}{175} & \cellcolor{gray!10}{3.15} & \cellcolor{gray!10}{3.440}\\
\hline
\end{tabular}}

}

\end{table}%

\begin{Shaded}
\begin{Highlighting}[]
\FunctionTok{kbl}\NormalTok{(dt, }
    \AttributeTok{caption =} \StringTok{"Stretch a small table to fit the page with the full width option."}\NormalTok{) }\SpecialCharTok{\%\textgreater{}\%}
  \FunctionTok{kable\_styling}\NormalTok{(}\AttributeTok{full\_width =}\NormalTok{ T, }\AttributeTok{latex\_options =} \FunctionTok{c}\NormalTok{(}\StringTok{"hold\_position"}\NormalTok{))}
\end{Highlighting}
\end{Shaded}

\begin{table}

\caption{\label{tbl-size-fullwidth}}

\centering{

[!h]
\centering
\caption{\label{tab:tbl-size-fullwidth}Stretch a small table to fit the page with the full width option.}
\centering
\begin{tabu} to \linewidth {>{\raggedright}X>{\raggedleft}X>{\raggedleft}X>{\raggedleft}X>{\raggedleft}X>{\raggedleft}X>{\raggedleft}X}
\hline
  & mpg & cyl & disp & hp & drat & wt\\
\hline
Mazda RX4 & 21.0 & 6 & 160 & 110 & 3.90 & 2.620\\
\hline
Mazda RX4 Wag & 21.0 & 6 & 160 & 110 & 3.90 & 2.875\\
\hline
Datsun 710 & 22.8 & 4 & 108 & 93 & 3.85 & 2.320\\
\hline
Hornet 4 Drive & 21.4 & 6 & 258 & 110 & 3.08 & 3.215\\
\hline
Hornet Sportabout & 18.7 & 8 & 360 & 175 & 3.15 & 3.440\\
\hline
\end{tabu}

}

\end{table}%

When you have very long tables that would extend over multiple pages,
use the \texttt{longtable} argument, usually in combination with the
\texttt{repeat\_header} option (Table~\ref{tbl-size-longtable}).

\begin{Shaded}
\begin{Highlighting}[]
\NormalTok{long\_dt }\OtherTok{\textless{}{-}} \FunctionTok{rbind}\NormalTok{(mtcars, mtcars) }\CommentTok{\# create a very long table}
\FunctionTok{kbl}\NormalTok{(long\_dt, }
    \AttributeTok{longtable =}\NormalTok{ T, }
    \AttributeTok{caption =} \StringTok{"Use the \textquotesingle{}longtable\textquotesingle{} argument in the kbl() to intelligently manage very long tables. Use the repeat header LaTeX option "}\NormalTok{) }\SpecialCharTok{\%\textgreater{}\%}
 \CommentTok{\# add\_header\_above(c(" ", "Group 1" = 5, "Group 2" = 6)) \%\textgreater{}\%}
  \FunctionTok{kable\_styling}\NormalTok{(}\AttributeTok{latex\_options =} \FunctionTok{c}\NormalTok{(}\StringTok{"repeat\_header"}\NormalTok{, }\StringTok{"hold\_position"}\NormalTok{))}
\end{Highlighting}
\end{Shaded}

\begin{longtable}[t]{l|r|r|r|r|r|r|r|r|r|r|r}

\caption{\label{tbl-size-longtable}Use the `longtable' argument in the
kbl() to intelligently manage very long tables. Use the repeat header
LaTeX option}

\tabularnewline

\\
\hline
  & mpg & cyl & disp & hp & drat & wt & qsec & vs & am & gear & carb\\
\hline
\endfirsthead
\caption[]{Use the 'longtable' argument in the kbl() to intelligently manage very long tables. Use the repeat header LaTeX option  \textit{(continued)}}\\
\hline
  & mpg & cyl & disp & hp & drat & wt & qsec & vs & am & gear & carb\\
\hline
\endhead
Mazda RX4 & 21.0 & 6 & 160.0 & 110 & 3.90 & 2.620 & 16.46 & 0 & 1 & 4 & 4\\
\hline
Mazda RX4 Wag & 21.0 & 6 & 160.0 & 110 & 3.90 & 2.875 & 17.02 & 0 & 1 & 4 & 4\\
\hline
Datsun 710 & 22.8 & 4 & 108.0 & 93 & 3.85 & 2.320 & 18.61 & 1 & 1 & 4 & 1\\
\hline
Hornet 4 Drive & 21.4 & 6 & 258.0 & 110 & 3.08 & 3.215 & 19.44 & 1 & 0 & 3 & 1\\
\hline
Hornet Sportabout & 18.7 & 8 & 360.0 & 175 & 3.15 & 3.440 & 17.02 & 0 & 0 & 3 & 2\\
\hline
Valiant & 18.1 & 6 & 225.0 & 105 & 2.76 & 3.460 & 20.22 & 1 & 0 & 3 & 1\\
\hline
Duster 360 & 14.3 & 8 & 360.0 & 245 & 3.21 & 3.570 & 15.84 & 0 & 0 & 3 & 4\\
\hline
Merc 240D & 24.4 & 4 & 146.7 & 62 & 3.69 & 3.190 & 20.00 & 1 & 0 & 4 & 2\\
\hline
Merc 230 & 22.8 & 4 & 140.8 & 95 & 3.92 & 3.150 & 22.90 & 1 & 0 & 4 & 2\\
\hline
Merc 280 & 19.2 & 6 & 167.6 & 123 & 3.92 & 3.440 & 18.30 & 1 & 0 & 4 & 4\\
\hline
Merc 280C & 17.8 & 6 & 167.6 & 123 & 3.92 & 3.440 & 18.90 & 1 & 0 & 4 & 4\\
\hline
Merc 450SE & 16.4 & 8 & 275.8 & 180 & 3.07 & 4.070 & 17.40 & 0 & 0 & 3 & 3\\
\hline
Merc 450SL & 17.3 & 8 & 275.8 & 180 & 3.07 & 3.730 & 17.60 & 0 & 0 & 3 & 3\\
\hline
Merc 450SLC & 15.2 & 8 & 275.8 & 180 & 3.07 & 3.780 & 18.00 & 0 & 0 & 3 & 3\\
\hline
Cadillac Fleetwood & 10.4 & 8 & 472.0 & 205 & 2.93 & 5.250 & 17.98 & 0 & 0 & 3 & 4\\
\hline
Lincoln Continental & 10.4 & 8 & 460.0 & 215 & 3.00 & 5.424 & 17.82 & 0 & 0 & 3 & 4\\
\hline
Chrysler Imperial & 14.7 & 8 & 440.0 & 230 & 3.23 & 5.345 & 17.42 & 0 & 0 & 3 & 4\\
\hline
Fiat 128 & 32.4 & 4 & 78.7 & 66 & 4.08 & 2.200 & 19.47 & 1 & 1 & 4 & 1\\
\hline
Honda Civic & 30.4 & 4 & 75.7 & 52 & 4.93 & 1.615 & 18.52 & 1 & 1 & 4 & 2\\
\hline
Toyota Corolla & 33.9 & 4 & 71.1 & 65 & 4.22 & 1.835 & 19.90 & 1 & 1 & 4 & 1\\
\hline
Toyota Corona & 21.5 & 4 & 120.1 & 97 & 3.70 & 2.465 & 20.01 & 1 & 0 & 3 & 1\\
\hline
Dodge Challenger & 15.5 & 8 & 318.0 & 150 & 2.76 & 3.520 & 16.87 & 0 & 0 & 3 & 2\\
\hline
AMC Javelin & 15.2 & 8 & 304.0 & 150 & 3.15 & 3.435 & 17.30 & 0 & 0 & 3 & 2\\
\hline
Camaro Z28 & 13.3 & 8 & 350.0 & 245 & 3.73 & 3.840 & 15.41 & 0 & 0 & 3 & 4\\
\hline
Pontiac Firebird & 19.2 & 8 & 400.0 & 175 & 3.08 & 3.845 & 17.05 & 0 & 0 & 3 & 2\\
\hline
Fiat X1-9 & 27.3 & 4 & 79.0 & 66 & 4.08 & 1.935 & 18.90 & 1 & 1 & 4 & 1\\
\hline
Porsche 914-2 & 26.0 & 4 & 120.3 & 91 & 4.43 & 2.140 & 16.70 & 0 & 1 & 5 & 2\\
\hline
Lotus Europa & 30.4 & 4 & 95.1 & 113 & 3.77 & 1.513 & 16.90 & 1 & 1 & 5 & 2\\
\hline
Ford Pantera L & 15.8 & 8 & 351.0 & 264 & 4.22 & 3.170 & 14.50 & 0 & 1 & 5 & 4\\
\hline
Ferrari Dino & 19.7 & 6 & 145.0 & 175 & 3.62 & 2.770 & 15.50 & 0 & 1 & 5 & 6\\
\hline
Maserati Bora & 15.0 & 8 & 301.0 & 335 & 3.54 & 3.570 & 14.60 & 0 & 1 & 5 & 8\\
\hline
Volvo 142E & 21.4 & 4 & 121.0 & 109 & 4.11 & 2.780 & 18.60 & 1 & 1 & 4 & 2\\
\hline
Mazda RX41 & 21.0 & 6 & 160.0 & 110 & 3.90 & 2.620 & 16.46 & 0 & 1 & 4 & 4\\
\hline
Mazda RX4 Wag1 & 21.0 & 6 & 160.0 & 110 & 3.90 & 2.875 & 17.02 & 0 & 1 & 4 & 4\\
\hline
Datsun 7101 & 22.8 & 4 & 108.0 & 93 & 3.85 & 2.320 & 18.61 & 1 & 1 & 4 & 1\\
\hline
Hornet 4 Drive1 & 21.4 & 6 & 258.0 & 110 & 3.08 & 3.215 & 19.44 & 1 & 0 & 3 & 1\\
\hline
Hornet Sportabout1 & 18.7 & 8 & 360.0 & 175 & 3.15 & 3.440 & 17.02 & 0 & 0 & 3 & 2\\
\hline
Valiant1 & 18.1 & 6 & 225.0 & 105 & 2.76 & 3.460 & 20.22 & 1 & 0 & 3 & 1\\
\hline
Duster 3601 & 14.3 & 8 & 360.0 & 245 & 3.21 & 3.570 & 15.84 & 0 & 0 & 3 & 4\\
\hline
Merc 240D1 & 24.4 & 4 & 146.7 & 62 & 3.69 & 3.190 & 20.00 & 1 & 0 & 4 & 2\\
\hline
Merc 2301 & 22.8 & 4 & 140.8 & 95 & 3.92 & 3.150 & 22.90 & 1 & 0 & 4 & 2\\
\hline
Merc 2801 & 19.2 & 6 & 167.6 & 123 & 3.92 & 3.440 & 18.30 & 1 & 0 & 4 & 4\\
\hline
Merc 280C1 & 17.8 & 6 & 167.6 & 123 & 3.92 & 3.440 & 18.90 & 1 & 0 & 4 & 4\\
\hline
Merc 450SE1 & 16.4 & 8 & 275.8 & 180 & 3.07 & 4.070 & 17.40 & 0 & 0 & 3 & 3\\
\hline
Merc 450SL1 & 17.3 & 8 & 275.8 & 180 & 3.07 & 3.730 & 17.60 & 0 & 0 & 3 & 3\\
\hline
Merc 450SLC1 & 15.2 & 8 & 275.8 & 180 & 3.07 & 3.780 & 18.00 & 0 & 0 & 3 & 3\\
\hline
Cadillac Fleetwood1 & 10.4 & 8 & 472.0 & 205 & 2.93 & 5.250 & 17.98 & 0 & 0 & 3 & 4\\
\hline
Lincoln Continental1 & 10.4 & 8 & 460.0 & 215 & 3.00 & 5.424 & 17.82 & 0 & 0 & 3 & 4\\
\hline
Chrysler Imperial1 & 14.7 & 8 & 440.0 & 230 & 3.23 & 5.345 & 17.42 & 0 & 0 & 3 & 4\\
\hline
Fiat 1281 & 32.4 & 4 & 78.7 & 66 & 4.08 & 2.200 & 19.47 & 1 & 1 & 4 & 1\\
\hline
Honda Civic1 & 30.4 & 4 & 75.7 & 52 & 4.93 & 1.615 & 18.52 & 1 & 1 & 4 & 2\\
\hline
Toyota Corolla1 & 33.9 & 4 & 71.1 & 65 & 4.22 & 1.835 & 19.90 & 1 & 1 & 4 & 1\\
\hline
Toyota Corona1 & 21.5 & 4 & 120.1 & 97 & 3.70 & 2.465 & 20.01 & 1 & 0 & 3 & 1\\
\hline
Dodge Challenger1 & 15.5 & 8 & 318.0 & 150 & 2.76 & 3.520 & 16.87 & 0 & 0 & 3 & 2\\
\hline
AMC Javelin1 & 15.2 & 8 & 304.0 & 150 & 3.15 & 3.435 & 17.30 & 0 & 0 & 3 & 2\\
\hline
Camaro Z281 & 13.3 & 8 & 350.0 & 245 & 3.73 & 3.840 & 15.41 & 0 & 0 & 3 & 4\\
\hline
Pontiac Firebird1 & 19.2 & 8 & 400.0 & 175 & 3.08 & 3.845 & 17.05 & 0 & 0 & 3 & 2\\
\hline
Fiat X1-91 & 27.3 & 4 & 79.0 & 66 & 4.08 & 1.935 & 18.90 & 1 & 1 & 4 & 1\\
\hline
Porsche 914-21 & 26.0 & 4 & 120.3 & 91 & 4.43 & 2.140 & 16.70 & 0 & 1 & 5 & 2\\
\hline
Lotus Europa1 & 30.4 & 4 & 95.1 & 113 & 3.77 & 1.513 & 16.90 & 1 & 1 & 5 & 2\\
\hline
Ford Pantera L1 & 15.8 & 8 & 351.0 & 264 & 4.22 & 3.170 & 14.50 & 0 & 1 & 5 & 4\\
\hline
Ferrari Dino1 & 19.7 & 6 & 145.0 & 175 & 3.62 & 2.770 & 15.50 & 0 & 1 & 5 & 6\\
\hline
Maserati Bora1 & 15.0 & 8 & 301.0 & 335 & 3.54 & 3.570 & 14.60 & 0 & 1 & 5 & 8\\
\hline
Volvo 142E1 & 21.4 & 4 & 121.0 & 109 & 4.11 & 2.780 & 18.60 & 1 & 1 & 4 & 2\\
\hline

\end{longtable}

By default, tables are aligned center on the page, but they can also be
aligned left or right. Note that alignment does not apply to full width
tables. Table~\ref{tbl-alignleft} uses the
\texttt{position\ =\ "left"\ argument} to align left. Oddly, aligning
right in the same way doesn't work as expected. In theory you can
``float right'' rather than position right, but this can produce errors.
It's unlikely you'll need to right align a table though.

\begin{Shaded}
\begin{Highlighting}[]
\FunctionTok{kbl}\NormalTok{(dt, }
    \AttributeTok{caption =} \StringTok{"Align your table (not the text, the whole table), \textquotesingle{}left\textquotesingle{} or \textquotesingle{}center\textquotesingle{} (the default) with the position argument."}\NormalTok{) }\SpecialCharTok{\%\textgreater{}\%} 
  \FunctionTok{kable\_styling}\NormalTok{(}\AttributeTok{position =} \StringTok{"left"}\NormalTok{, }\AttributeTok{latex\_options =} \FunctionTok{c}\NormalTok{(}\StringTok{"hold\_position"}\NormalTok{))}
\end{Highlighting}
\end{Shaded}

\begin{table}

\caption{\label{tbl-alignleft}}

\centering{

[!h]

\caption{Align your table (not the text, the whole table), 'left' or 'center' (the default) with the position argument.}
\begin{tabular}[t]{l|r|r|r|r|r|r}
\hline
  & mpg & cyl & disp & hp & drat & wt\\
\hline
Mazda RX4 & 21.0 & 6 & 160 & 110 & 3.90 & 2.620\\
\hline
Mazda RX4 Wag & 21.0 & 6 & 160 & 110 & 3.90 & 2.875\\
\hline
Datsun 710 & 22.8 & 4 & 108 & 93 & 3.85 & 2.320\\
\hline
Hornet 4 Drive & 21.4 & 6 & 258 & 110 & 3.08 & 3.215\\
\hline
Hornet Sportabout & 18.7 & 8 & 360 & 175 & 3.15 & 3.440\\
\hline
\end{tabular}

}

\end{table}%

You can make style adjustments to the whole table (e.g., font fact,
background color, bold/italic) or to specific rows and columns.
Table~\ref{tbl-col-specs} includes three \texttt{column\_spec} functions
to modify properties in columns 2, 6, and 9.

\begin{Shaded}
\begin{Highlighting}[]
\NormalTok{mtcars[}\DecValTok{1}\SpecialCharTok{:}\DecValTok{8}\NormalTok{, }\DecValTok{1}\SpecialCharTok{:}\DecValTok{8}\NormalTok{] }\SpecialCharTok{\%\textgreater{}\%}
\FunctionTok{kbl}\NormalTok{(}\AttributeTok{booktabs =}\NormalTok{ T,}
    \AttributeTok{caption =} \StringTok{"This table uses column specifications to make many visual modifications to individual rows. The final line shows how the same function can modify just one cell in a column in a similar way."}\NormalTok{) }\SpecialCharTok{\%\textgreater{}\%}
  \FunctionTok{kable\_paper}\NormalTok{(}\AttributeTok{full\_width =}\NormalTok{ F) }\SpecialCharTok{\%\textgreater{}\%}
  \FunctionTok{column\_spec}\NormalTok{(}\DecValTok{2}\NormalTok{, }\AttributeTok{color =} \FunctionTok{spec\_color}\NormalTok{(mtcars}\SpecialCharTok{$}\NormalTok{mpg[}\DecValTok{1}\SpecialCharTok{:}\DecValTok{8}\NormalTok{]),}
             \AttributeTok{link =} \StringTok{"https://haozhu233.github.io/kableExtra"}\NormalTok{) }\SpecialCharTok{\%\textgreater{}\%}
  \FunctionTok{column\_spec}\NormalTok{(}\DecValTok{6}\NormalTok{, }\AttributeTok{color =} \StringTok{"white"}\NormalTok{,}
             \AttributeTok{background =} \FunctionTok{spec\_color}\NormalTok{(mtcars}\SpecialCharTok{$}\NormalTok{drat[}\DecValTok{1}\SpecialCharTok{:}\DecValTok{8}\NormalTok{], }\AttributeTok{end =} \FloatTok{0.7}\NormalTok{),}
             \AttributeTok{popover =} \FunctionTok{paste}\NormalTok{(}\StringTok{"am:"}\NormalTok{, mtcars}\SpecialCharTok{$}\NormalTok{am[}\DecValTok{1}\SpecialCharTok{:}\DecValTok{8}\NormalTok{])) }\SpecialCharTok{\%\textgreater{}\%}
\CommentTok{\#  The c(rep(XX,7), XX) bit here is basically saying}
\CommentTok{\#  in this column (9) treat the first 7 columns one way and then the rest (the last)}
\CommentTok{\#  this other way. e.g., strikeout should be FALSE for the 1st seven rows but TRUE}
\CommentTok{\#  for the rest, color should be BLACK for 1{-}7 but RED for the rest}
  \FunctionTok{column\_spec}\NormalTok{(}\DecValTok{9}\NormalTok{, }\AttributeTok{strikeout =} \FunctionTok{c}\NormalTok{(}\FunctionTok{rep}\NormalTok{(F, }\DecValTok{7}\NormalTok{), T), }\AttributeTok{bold =} \FunctionTok{c}\NormalTok{(}\FunctionTok{rep}\NormalTok{(F, }\DecValTok{7}\NormalTok{), T),}
             \AttributeTok{color =} \FunctionTok{c}\NormalTok{(}\FunctionTok{rep}\NormalTok{(}\StringTok{"black"}\NormalTok{, }\DecValTok{7}\NormalTok{), }\StringTok{"red"}\NormalTok{))}
\end{Highlighting}
\end{Shaded}

\begin{table}

\caption{\label{tbl-col-specs}}

\centering{

\centering
\caption{This table uses column specifications to make many visual modifications to individual rows. The final line shows how the same function can modify just one cell in a column in a similar way.}
\centering
\begin{tabular}[t]{l>{}rrrr>{}rrr>{}r}
\toprule
  & mpg & cyl & disp & hp & drat & wt & qsec & vs\\
\midrule
Mazda RX4 & \href{https://haozhu233.github.io/kableExtra}{\textcolor[HTML]{34B679}{21.0}} & 6 & 160.0 & 110 & \cellcolor[HTML]{43BF71}{\textcolor{white}{3.90}} & 2.620 & 16.46 & \textcolor{black}{0}\\
Mazda RX4 Wag & \href{https://haozhu233.github.io/kableExtra}{\textcolor[HTML]{34B679}{21.0}} & 6 & 160.0 & 110 & \cellcolor[HTML]{43BF71}{\textcolor{white}{3.90}} & 2.875 & 17.02 & \textcolor{black}{0}\\
Datsun 710 & \href{https://haozhu233.github.io/kableExtra}{\textcolor[HTML]{95D840}{22.8}} & 4 & 108.0 & 93 & \cellcolor[HTML]{37B878}{\textcolor{white}{3.85}} & 2.320 & 18.61 & \textcolor{black}{1}\\
Hornet 4 Drive & \href{https://haozhu233.github.io/kableExtra}{\textcolor[HTML]{44BF70}{21.4}} & 6 & 258.0 & 110 & \cellcolor[HTML]{414387}{\textcolor{white}{3.08}} & 3.215 & 19.44 & \textcolor{black}{1}\\
Hornet Sportabout & \href{https://haozhu233.github.io/kableExtra}{\textcolor[HTML]{26818E}{18.7}} & 8 & 360.0 & 175 & \cellcolor[HTML]{3C4F8A}{\textcolor{white}{3.15}} & 3.440 & 17.02 & \textcolor{black}{0}\\
\addlinespace
Valiant & \href{https://haozhu233.github.io/kableExtra}{\textcolor[HTML]{2C728E}{18.1}} & 6 & 225.0 & 105 & \cellcolor[HTML]{440154}{\textcolor{white}{2.76}} & 3.460 & 20.22 & \textcolor{black}{1}\\
Duster 360 & \href{https://haozhu233.github.io/kableExtra}{\textcolor[HTML]{440154}{14.3}} & 8 & 360.0 & 245 & \cellcolor[HTML]{375A8C}{\textcolor{white}{3.21}} & 3.570 & 15.84 & \textcolor{black}{0}\\
Merc 240D & \href{https://haozhu233.github.io/kableExtra}{\textcolor[HTML]{FDE725}{24.4}} & 4 & 146.7 & 62 & \cellcolor[HTML]{1FA187}{\textcolor{white}{3.69}} & 3.190 & 20.00 & \textcolor{red}{\sout{\textbf{1}}}\\
\bottomrule
\end{tabular}

}

\end{table}%

Table~\ref{tbl-row-specs} includes two \texttt{row\_spec} functions to
modify properties in rows 1 and rows 3-5.

\begin{Shaded}
\begin{Highlighting}[]
\FunctionTok{kbl}\NormalTok{(dt, }\AttributeTok{booktabs =}\NormalTok{ T,}
    \AttributeTok{caption =} \StringTok{"You can specify styling for rows in the same way."}\NormalTok{) }\SpecialCharTok{\%\textgreater{}\%}
  \FunctionTok{kable\_styling}\NormalTok{(}\StringTok{"striped"}\NormalTok{, }\AttributeTok{full\_width =}\NormalTok{ F) }\SpecialCharTok{\%\textgreater{}\%}
  \FunctionTok{column\_spec}\NormalTok{(}\DecValTok{7}\NormalTok{, }\AttributeTok{border\_left =}\NormalTok{ T, }\AttributeTok{bold =}\NormalTok{ T) }\SpecialCharTok{\%\textgreater{}\%}
  \FunctionTok{row\_spec}\NormalTok{(}\DecValTok{1}\NormalTok{, }\AttributeTok{strikeout =}\NormalTok{ T) }\SpecialCharTok{\%\textgreater{}\%}
  \FunctionTok{row\_spec}\NormalTok{(}\DecValTok{3}\SpecialCharTok{:}\DecValTok{5}\NormalTok{, }\AttributeTok{bold =}\NormalTok{ T, }\AttributeTok{color =} \StringTok{"white"}\NormalTok{, }\AttributeTok{background =} \StringTok{"black"}\NormalTok{)}
\end{Highlighting}
\end{Shaded}

\begin{table}

\caption{\label{tbl-row-specs}}

\centering{

\centering
\caption{You can specify styling for rows in the same way.}
\centering
\begin{tabular}[t]{lrrrrr|>{}r}
\toprule
  & mpg & cyl & disp & hp & drat & wt\\
\midrule
\sout{Mazda RX4} & \sout{21.0} & \sout{6} & \sout{160} & \sout{110} & \sout{3.90} & \sout{\textbf{2.620}}\\
Mazda RX4 Wag & 21.0 & 6 & 160 & 110 & 3.90 & \textbf{2.875}\\
\cellcolor{black}{\textcolor{white}{\textbf{Datsun 710}}} & \cellcolor{black}{\textcolor{white}{\textbf{22.8}}} & \cellcolor{black}{\textcolor{white}{\textbf{4}}} & \cellcolor{black}{\textcolor{white}{\textbf{108}}} & \cellcolor{black}{\textcolor{white}{\textbf{93}}} & \cellcolor{black}{\textcolor{white}{\textbf{3.85}}} & \cellcolor{black}{\textcolor{white}{\textbf{\textbf{2.320}}}}\\
\cellcolor{black}{\textcolor{white}{\textbf{Hornet 4 Drive}}} & \cellcolor{black}{\textcolor{white}{\textbf{21.4}}} & \cellcolor{black}{\textcolor{white}{\textbf{6}}} & \cellcolor{black}{\textcolor{white}{\textbf{258}}} & \cellcolor{black}{\textcolor{white}{\textbf{110}}} & \cellcolor{black}{\textcolor{white}{\textbf{3.08}}} & \cellcolor{black}{\textcolor{white}{\textbf{\textbf{3.215}}}}\\
\cellcolor{black}{\textcolor{white}{\textbf{Hornet Sportabout}}} & \cellcolor{black}{\textcolor{white}{\textbf{18.7}}} & \cellcolor{black}{\textcolor{white}{\textbf{8}}} & \cellcolor{black}{\textcolor{white}{\textbf{360}}} & \cellcolor{black}{\textcolor{white}{\textbf{175}}} & \cellcolor{black}{\textcolor{white}{\textbf{3.15}}} & \cellcolor{black}{\textcolor{white}{\textbf{\textbf{3.440}}}}\\
\bottomrule
\end{tabular}

}

\end{table}%

You can ``group'' rows and columns, basically equivalent to merging
cells in a spreadsheet. This is commonly needed for multi-level headers
or organizing non-rectangular data (like a lot of model output).
Table~\ref{tbl-grouping-addheader} uses the
\texttt{add\_header\_above()} function to do so.

\begin{Shaded}
\begin{Highlighting}[]
\FunctionTok{kbl}\NormalTok{(dt, }
    \AttributeTok{caption =} \StringTok{"Create grouped columns by adding a header."}\NormalTok{) }\SpecialCharTok{\%\textgreater{}\%}
  \FunctionTok{kable\_styling}\NormalTok{(}\AttributeTok{latex\_options =} \FunctionTok{c}\NormalTok{(}\StringTok{"striped"}\NormalTok{, }\StringTok{"hold\_position"}\NormalTok{)) }\SpecialCharTok{\%\textgreater{}\%} 
  \CommentTok{\# The syntax within the list is "name of group header = [number of columns to be grouped under it]. Make sure the total number of columns to group equal the actual number of columns in the table}
  \FunctionTok{add\_header\_above}\NormalTok{(}\FunctionTok{c}\NormalTok{(}\StringTok{" "} \OtherTok{=} \DecValTok{1}\NormalTok{, }\StringTok{"Group 1"} \OtherTok{=} \DecValTok{2}\NormalTok{, }\StringTok{"Group 2"} \OtherTok{=} \DecValTok{2}\NormalTok{, }\StringTok{"Group 3"} \OtherTok{=} \DecValTok{2}\NormalTok{))}
\end{Highlighting}
\end{Shaded}

\begin{table}

\caption{\label{tbl-grouping-addheader}}

\centering{

[!h]
\centering
\caption{Create grouped columns by adding a header.}
\centering
\begin{tabular}[t]{l|r|r|r|r|r|r}
\hline
\multicolumn{1}{c|}{ } & \multicolumn{2}{c|}{Group 1} & \multicolumn{2}{c|}{Group 2} & \multicolumn{2}{c}{Group 3} \\
\cline{2-3} \cline{4-5} \cline{6-7}
  & mpg & cyl & disp & hp & drat & wt\\
\hline
\cellcolor{gray!10}{Mazda RX4} & \cellcolor{gray!10}{21.0} & \cellcolor{gray!10}{6} & \cellcolor{gray!10}{160} & \cellcolor{gray!10}{110} & \cellcolor{gray!10}{3.90} & \cellcolor{gray!10}{2.620}\\
\hline
Mazda RX4 Wag & 21.0 & 6 & 160 & 110 & 3.90 & 2.875\\
\hline
\cellcolor{gray!10}{Datsun 710} & \cellcolor{gray!10}{22.8} & \cellcolor{gray!10}{4} & \cellcolor{gray!10}{108} & \cellcolor{gray!10}{93} & \cellcolor{gray!10}{3.85} & \cellcolor{gray!10}{2.320}\\
\hline
Hornet 4 Drive & 21.4 & 6 & 258 & 110 & 3.08 & 3.215\\
\hline
\cellcolor{gray!10}{Hornet Sportabout} & \cellcolor{gray!10}{18.7} & \cellcolor{gray!10}{8} & \cellcolor{gray!10}{360} & \cellcolor{gray!10}{175} & \cellcolor{gray!10}{3.15} & \cellcolor{gray!10}{3.440}\\
\hline
\end{tabular}

}

\end{table}%

You can group more than once and format these header rows individually.
Use the total number of rows to specify how many columns to group, not
the number of rows in the grouped row below it
(Table~\ref{tbl-grouping-multiple}).

\begin{Shaded}
\begin{Highlighting}[]
\FunctionTok{kbl}\NormalTok{(dt, }
    \AttributeTok{caption =} \StringTok{"Group multiple times and format these header rows individually."}\NormalTok{) }\SpecialCharTok{\%\textgreater{}\%}
  \FunctionTok{kable\_styling}\NormalTok{(}\AttributeTok{latex\_options =} \FunctionTok{c}\NormalTok{(}\StringTok{"striped"}\NormalTok{, }\StringTok{"hold\_position"}\NormalTok{)) }\SpecialCharTok{\%\textgreater{}\%} 
  \CommentTok{\# If you don\textquotesingle{}t specify how many columns to group for that group name, it assumes =1}
  \FunctionTok{add\_header\_above}\NormalTok{(}\FunctionTok{c}\NormalTok{(}\StringTok{" "}\NormalTok{, }\StringTok{"Group 1"} \OtherTok{=} \DecValTok{2}\NormalTok{, }\StringTok{"Group 2"} \OtherTok{=} \DecValTok{2}\NormalTok{, }\StringTok{"Group 3"} \OtherTok{=} \DecValTok{2}\NormalTok{)) }\SpecialCharTok{\%\textgreater{}\%}
  \FunctionTok{add\_header\_above}\NormalTok{(}\FunctionTok{c}\NormalTok{(}\StringTok{" "}\NormalTok{, }\StringTok{"Group 4"} \OtherTok{=} \DecValTok{4}\NormalTok{, }\StringTok{"Group 5"} \OtherTok{=} \DecValTok{2}\NormalTok{)) }\SpecialCharTok{\%\textgreater{}\%}
  \FunctionTok{add\_header\_above}\NormalTok{(}\FunctionTok{c}\NormalTok{(}\StringTok{" "}\NormalTok{, }\StringTok{"Group 6"} \OtherTok{=} \DecValTok{6}\NormalTok{), }\AttributeTok{bold =}\NormalTok{ T, }\AttributeTok{italic =}\NormalTok{ T)}
\end{Highlighting}
\end{Shaded}

\begin{table}

\caption{\label{tbl-grouping-multiple}}

\centering{

[!h]
\centering
\caption{Group multiple times and format these header rows individually.}
\centering
\begin{tabular}[t]{l|r|r|r|r|r|r}
\hline
\multicolumn{1}{c|}{\em{\textbf{ }}} & \multicolumn{6}{c}{\em{\textbf{Group 6}}} \\
\cline{2-7}
\multicolumn{1}{c|}{ } & \multicolumn{4}{c|}{Group 4} & \multicolumn{2}{c}{Group 5} \\
\cline{2-5} \cline{6-7}
\multicolumn{1}{c|}{ } & \multicolumn{2}{c|}{Group 1} & \multicolumn{2}{c|}{Group 2} & \multicolumn{2}{c}{Group 3} \\
\cline{2-3} \cline{4-5} \cline{6-7}
  & mpg & cyl & disp & hp & drat & wt\\
\hline
\cellcolor{gray!10}{Mazda RX4} & \cellcolor{gray!10}{21.0} & \cellcolor{gray!10}{6} & \cellcolor{gray!10}{160} & \cellcolor{gray!10}{110} & \cellcolor{gray!10}{3.90} & \cellcolor{gray!10}{2.620}\\
\hline
Mazda RX4 Wag & 21.0 & 6 & 160 & 110 & 3.90 & 2.875\\
\hline
\cellcolor{gray!10}{Datsun 710} & \cellcolor{gray!10}{22.8} & \cellcolor{gray!10}{4} & \cellcolor{gray!10}{108} & \cellcolor{gray!10}{93} & \cellcolor{gray!10}{3.85} & \cellcolor{gray!10}{2.320}\\
\hline
Hornet 4 Drive & 21.4 & 6 & 258 & 110 & 3.08 & 3.215\\
\hline
\cellcolor{gray!10}{Hornet Sportabout} & \cellcolor{gray!10}{18.7} & \cellcolor{gray!10}{8} & \cellcolor{gray!10}{360} & \cellcolor{gray!10}{175} & \cellcolor{gray!10}{3.15} & \cellcolor{gray!10}{3.440}\\
\hline
\end{tabular}

}

\end{table}%

Grouping rows is a little different. Table~\ref{tbl-grouping-packrows}
uses the \texttt{pack\_rows()} function to effectively add ``header
rows'' throughout the table.

\begin{Shaded}
\begin{Highlighting}[]
\FunctionTok{kbl}\NormalTok{(mtcars[}\DecValTok{1}\SpecialCharTok{:}\DecValTok{10}\NormalTok{, }\DecValTok{1}\SpecialCharTok{:}\DecValTok{6}\NormalTok{], }
    \AttributeTok{caption =} \StringTok{"Group rows using packing."}\NormalTok{) }\SpecialCharTok{\%\textgreater{}\%}
  \FunctionTok{kable\_styling}\NormalTok{(}\AttributeTok{latex\_options =} \FunctionTok{c}\NormalTok{(}\StringTok{"striped"}\NormalTok{, }\StringTok{"hold\_position"}\NormalTok{)) }\SpecialCharTok{\%\textgreater{}\%} 
  \CommentTok{\# syntax here is ("group label", [int of first row to include in group][int of last row to include])}
  \FunctionTok{pack\_rows}\NormalTok{(}\StringTok{"Group 1"}\NormalTok{, }\DecValTok{4}\NormalTok{, }\DecValTok{7}\NormalTok{) }\SpecialCharTok{\%\textgreater{}\%}
  \FunctionTok{pack\_rows}\NormalTok{(}\StringTok{"Group 2"}\NormalTok{, }\DecValTok{8}\NormalTok{, }\DecValTok{10}\NormalTok{)}
\NormalTok{collapse\_rows\_dt }\OtherTok{\textless{}{-}} \FunctionTok{data.frame}\NormalTok{(}\AttributeTok{C1 =} \FunctionTok{c}\NormalTok{(}\FunctionTok{rep}\NormalTok{(}\StringTok{"a"}\NormalTok{, }\DecValTok{10}\NormalTok{), }\FunctionTok{rep}\NormalTok{(}\StringTok{"b"}\NormalTok{, }\DecValTok{5}\NormalTok{)),}
                               \AttributeTok{C2 =} \FunctionTok{c}\NormalTok{(}\FunctionTok{rep}\NormalTok{(}\StringTok{"c"}\NormalTok{, }\DecValTok{7}\NormalTok{), }\FunctionTok{rep}\NormalTok{(}\StringTok{"d"}\NormalTok{, }\DecValTok{3}\NormalTok{), }\FunctionTok{rep}\NormalTok{(}\StringTok{"c"}\NormalTok{, }\DecValTok{2}\NormalTok{), }\FunctionTok{rep}\NormalTok{(}\StringTok{"d"}\NormalTok{, }\DecValTok{3}\NormalTok{)),}
                               \AttributeTok{C3 =} \DecValTok{1}\SpecialCharTok{:}\DecValTok{15}\NormalTok{,}
                               \AttributeTok{C4 =} \FunctionTok{sample}\NormalTok{(}\FunctionTok{c}\NormalTok{(}\DecValTok{0}\NormalTok{,}\DecValTok{1}\NormalTok{), }\DecValTok{15}\NormalTok{, }\AttributeTok{replace =} \ConstantTok{TRUE}\NormalTok{))}
\end{Highlighting}
\end{Shaded}

\begin{table}

\caption{\label{tbl-grouping-packrows}}

\centering{

[!h]
\centering
\caption{Group rows using packing.}
\centering
\begin{tabular}[t]{l|r|r|r|r|r|r}
\hline
  & mpg & cyl & disp & hp & drat & wt\\
\hline
\cellcolor{gray!10}{Mazda RX4} & \cellcolor{gray!10}{21.0} & \cellcolor{gray!10}{6} & \cellcolor{gray!10}{160.0} & \cellcolor{gray!10}{110} & \cellcolor{gray!10}{3.90} & \cellcolor{gray!10}{2.620}\\
\hline
Mazda RX4 Wag & 21.0 & 6 & 160.0 & 110 & 3.90 & 2.875\\
\hline
\cellcolor{gray!10}{Datsun 710} & \cellcolor{gray!10}{22.8} & \cellcolor{gray!10}{4} & \cellcolor{gray!10}{108.0} & \cellcolor{gray!10}{93} & \cellcolor{gray!10}{3.85} & \cellcolor{gray!10}{2.320}\\
\hline
\multicolumn{7}{l}{\textbf{Group 1}}\\
\hline
\hspace{1em}Hornet 4 Drive & 21.4 & 6 & 258.0 & 110 & 3.08 & 3.215\\
\hline
\hspace{1em}\cellcolor{gray!10}{Hornet Sportabout} & \cellcolor{gray!10}{18.7} & \cellcolor{gray!10}{8} & \cellcolor{gray!10}{360.0} & \cellcolor{gray!10}{175} & \cellcolor{gray!10}{3.15} & \cellcolor{gray!10}{3.440}\\
\hline
\hspace{1em}Valiant & 18.1 & 6 & 225.0 & 105 & 2.76 & 3.460\\
\hline
\hspace{1em}\cellcolor{gray!10}{Duster 360} & \cellcolor{gray!10}{14.3} & \cellcolor{gray!10}{8} & \cellcolor{gray!10}{360.0} & \cellcolor{gray!10}{245} & \cellcolor{gray!10}{3.21} & \cellcolor{gray!10}{3.570}\\
\hline
\multicolumn{7}{l}{\textbf{Group 2}}\\
\hline
\hspace{1em}Merc 240D & 24.4 & 4 & 146.7 & 62 & 3.69 & 3.190\\
\hline
\hspace{1em}\cellcolor{gray!10}{Merc 230} & \cellcolor{gray!10}{22.8} & \cellcolor{gray!10}{4} & \cellcolor{gray!10}{140.8} & \cellcolor{gray!10}{95} & \cellcolor{gray!10}{3.92} & \cellcolor{gray!10}{3.150}\\
\hline
\hspace{1em}Merc 280 & 19.2 & 6 & 167.6 & 123 & 3.92 & 3.440\\
\hline
\end{tabular}

}

\end{table}%

Alternatively, use \texttt{collapse\_rows()} in
Table~\ref{tbl-grouping-collapse} for more of a ``merge cells'' effect.

\begin{Shaded}
\begin{Highlighting}[]
\NormalTok{collapse\_rows\_dt }\OtherTok{\textless{}{-}} \FunctionTok{data.frame}\NormalTok{(}\AttributeTok{C1 =} \FunctionTok{c}\NormalTok{(}\FunctionTok{rep}\NormalTok{(}\StringTok{"a"}\NormalTok{, }\DecValTok{10}\NormalTok{), }\FunctionTok{rep}\NormalTok{(}\StringTok{"b"}\NormalTok{, }\DecValTok{5}\NormalTok{)),}
                               \AttributeTok{C2 =} \FunctionTok{c}\NormalTok{(}\FunctionTok{rep}\NormalTok{(}\StringTok{"c"}\NormalTok{, }\DecValTok{7}\NormalTok{), }\FunctionTok{rep}\NormalTok{(}\StringTok{"d"}\NormalTok{, }\DecValTok{3}\NormalTok{), }\FunctionTok{rep}\NormalTok{(}\StringTok{"c"}\NormalTok{, }\DecValTok{2}\NormalTok{), }\FunctionTok{rep}\NormalTok{(}\StringTok{"d"}\NormalTok{, }\DecValTok{3}\NormalTok{)),}
                               \AttributeTok{C3 =} \DecValTok{1}\SpecialCharTok{:}\DecValTok{15}\NormalTok{,}
                               \AttributeTok{C4 =} \FunctionTok{sample}\NormalTok{(}\FunctionTok{c}\NormalTok{(}\DecValTok{0}\NormalTok{,}\DecValTok{1}\NormalTok{), }\DecValTok{15}\NormalTok{, }\AttributeTok{replace =} \ConstantTok{TRUE}\NormalTok{))}

\CommentTok{\# You will only see the "collapse" happen when you knit}
\CommentTok{\# not when you run the chunk}
\FunctionTok{kbl}\NormalTok{(collapse\_rows\_dt, }\AttributeTok{booktabs =}\NormalTok{ T, }\AttributeTok{align =} \StringTok{"c"}\NormalTok{,}
    \AttributeTok{caption =} \StringTok{"Collapse rows for more of a \textquotesingle{}merge cells\textquotesingle{} effect."}\NormalTok{) }\SpecialCharTok{\%\textgreater{}\%}
  \FunctionTok{column\_spec}\NormalTok{(}\DecValTok{1}\NormalTok{, }\AttributeTok{bold=}\NormalTok{T) }\SpecialCharTok{\%\textgreater{}\%}
  \FunctionTok{collapse\_rows}\NormalTok{(}\AttributeTok{columns =} \DecValTok{1}\SpecialCharTok{:}\DecValTok{2}\NormalTok{,}
                \AttributeTok{latex\_hline =} \StringTok{"major"}\NormalTok{,}
                \AttributeTok{row\_group\_label\_position =} \StringTok{"first"}\NormalTok{)}
\end{Highlighting}
\end{Shaded}

\begin{table}

\caption{\label{tbl-grouping-collapse}}

\centering{

\caption{Collapse rows for more of a 'merge cells' effect.}
\centering
\begin{tabular}[t]{>{}cccc}
\toprule
C1 & C2 & C3 & C4\\
\midrule
\textbf{a} & c & 1 & 0\\

 &  & 2 & 1\\

 &  & 3 & 0\\

 &  & 4 & 0\\

 &  & 5 & 0\\

 &  & 6 & 0\\

 &  & 7 & 1\\

 & d & 8 & 1\\

 &  & 9 & 1\\

 &  & 10 & 1\\
\cmidrule{1-4}
\textbf{b} & c & 11 & 0\\

 &  & 12 & 1\\

 & d & 13 & 0\\

 &  & 14 & 1\\

 &  & 15 & 1\\
\bottomrule
\end{tabular}

}

\end{table}%

Finally, you can add notes or footnotes below your table with the
\texttt{footnote()} function. Table~\ref{tbl-footnotes} includes the
`general' argument for an unlabeled note and `number', `alphabet', and
`symbol' for footnotes with ordered labels.

\begin{Shaded}
\begin{Highlighting}[]
\FunctionTok{kbl}\NormalTok{(dt, }\AttributeTok{align =} \StringTok{"c"}\NormalTok{,}
    \AttributeTok{caption =} \StringTok{"Add unordered (general) and ordered footnotes."}\NormalTok{) }\SpecialCharTok{\%\textgreater{}\%}
  \FunctionTok{kable\_styling}\NormalTok{(}\AttributeTok{full\_width =}\NormalTok{ F) }\SpecialCharTok{\%\textgreater{}\%}
  \FunctionTok{footnote}\NormalTok{(}\AttributeTok{general =} \StringTok{"Here is a general comments of the table. "}\NormalTok{,}
           \AttributeTok{number =} \FunctionTok{c}\NormalTok{(}\StringTok{"Footnote 1; "}\NormalTok{, }\StringTok{"Footnote 2; "}\NormalTok{),}
           \AttributeTok{alphabet =} \FunctionTok{c}\NormalTok{(}\StringTok{"Footnote A; "}\NormalTok{, }\StringTok{"Footnote B; "}\NormalTok{),}
           \AttributeTok{symbol =} \FunctionTok{c}\NormalTok{(}\StringTok{"Footnote Symbol 1; "}\NormalTok{, }\StringTok{"Footnote Symbol 2"}\NormalTok{)}
\NormalTok{)}
\end{Highlighting}
\end{Shaded}

\begin{table}

\caption{\label{tbl-footnotes}}

\centering{

\centering
\caption{Add unordered (general) and ordered footnotes.}
\centering
\begin{tabular}[t]{l|c|c|c|c|c|c}
\hline
  & mpg & cyl & disp & hp & drat & wt\\
\hline
Mazda RX4 & 21.0 & 6 & 160 & 110 & 3.90 & 2.620\\
\hline
Mazda RX4 Wag & 21.0 & 6 & 160 & 110 & 3.90 & 2.875\\
\hline
Datsun 710 & 22.8 & 4 & 108 & 93 & 3.85 & 2.320\\
\hline
Hornet 4 Drive & 21.4 & 6 & 258 & 110 & 3.08 & 3.215\\
\hline
Hornet Sportabout & 18.7 & 8 & 360 & 175 & 3.15 & 3.440\\
\hline
\multicolumn{7}{l}{\rule{0pt}{1em}\textit{Note: }}\\
\multicolumn{7}{l}{\rule{0pt}{1em}Here is a general comments of the table. }\\
\multicolumn{7}{l}{\rule{0pt}{1em}\textsuperscript{1} Footnote 1; }\\
\multicolumn{7}{l}{\rule{0pt}{1em}\textsuperscript{2} Footnote 2; }\\
\multicolumn{7}{l}{\rule{0pt}{1em}\textsuperscript{a} Footnote A; }\\
\multicolumn{7}{l}{\rule{0pt}{1em}\textsuperscript{b} Footnote B; }\\
\multicolumn{7}{l}{\rule{0pt}{1em}\textsuperscript{*} Footnote Symbol 1; }\\
\multicolumn{7}{l}{\rule{0pt}{1em}\textsuperscript{\dag} Footnote Symbol 2}\\
\end{tabular}

}

\end{table}%

There is \textbf{so much more} you can do with kables! I am personally a
huge fan of kables and the kableExtra package because it works
beautifully with papaja, but you can also explore other packages for
making beautiful tables:

\begin{itemize}
\tightlist
\item
  \href{https://davidgohel.github.io/flextable/}{flextable}
\item
  \href{https://gt.rstudio.com/}{gt table}
\item
  \href{http://xtable.r-forge.r-project.org/}{xtable}
\item
  \href{https://cran.r-project.org/web/packages/stargazer/index.html}{stargazer}

  \begin{itemize}
  \tightlist
  \item
    This one is designed to simplify tables for regression output
  \end{itemize}
\end{itemize}

\subsection{Table: flextable}\label{table-flextable}

\begin{Shaded}
\begin{Highlighting}[]
\FunctionTok{library}\NormalTok{(flextable)}
\end{Highlighting}
\end{Shaded}

\begin{verbatim}

Attaching package: 'flextable'
\end{verbatim}

\begin{verbatim}
The following objects are masked from 'package:kableExtra':

    as_image, footnote
\end{verbatim}

\begin{verbatim}
The following object is masked from 'package:purrr':

    compose
\end{verbatim}

\begin{Shaded}
\begin{Highlighting}[]
\FunctionTok{tibble}\NormalTok{(}\AttributeTok{Letters =} \FunctionTok{c}\NormalTok{(}\StringTok{"A"}\NormalTok{, }\StringTok{"B"}\NormalTok{, }\StringTok{"C"}\NormalTok{), }
       \AttributeTok{Numbers =} \DecValTok{1}\SpecialCharTok{:}\DecValTok{3}\NormalTok{) }\SpecialCharTok{\%\textgreater{}\%} 
  \FunctionTok{flextable}\NormalTok{() }\SpecialCharTok{\%\textgreater{}\%} 
  \FunctionTok{theme\_apa}\NormalTok{() }
\end{Highlighting}
\end{Shaded}

\global\setlength{\Oldarrayrulewidth}{\arrayrulewidth}

\global\setlength{\Oldtabcolsep}{\tabcolsep}

\setlength{\tabcolsep}{2pt}

\renewcommand*{\arraystretch}{1.5}



\providecommand{\ascline}[3]{\noalign{\global\arrayrulewidth #1}\arrayrulecolor[HTML]{#2}\cline{#3}}

\begin{longtable}[l]{|p{0.75in}|p{0.75in}}

\caption{\label{tbl-mytable}My Table}

\tabularnewline

\ascline{0.75pt}{000000}{1-2}

\multicolumn{1}{>{\centering}m{\dimexpr 0.75in+0\tabcolsep}}{\textcolor[HTML]{000000}{\fontsize{11}{22}\selectfont{\global\setmainfont{Times New Roman}{Letters}}}} & \multicolumn{1}{>{\centering}m{\dimexpr 0.75in+0\tabcolsep}}{\textcolor[HTML]{000000}{\fontsize{11}{22}\selectfont{\global\setmainfont{Times New Roman}{Numbers}}}} \\

\ascline{0.75pt}{000000}{1-2}\endfirsthead 

\ascline{0.75pt}{000000}{1-2}

\multicolumn{1}{>{\centering}m{\dimexpr 0.75in+0\tabcolsep}}{\textcolor[HTML]{000000}{\fontsize{11}{22}\selectfont{\global\setmainfont{Times New Roman}{Letters}}}} & \multicolumn{1}{>{\centering}m{\dimexpr 0.75in+0\tabcolsep}}{\textcolor[HTML]{000000}{\fontsize{11}{22}\selectfont{\global\setmainfont{Times New Roman}{Numbers}}}} \\

\ascline{0.75pt}{000000}{1-2}\endhead



\multicolumn{1}{>{\centering}m{\dimexpr 0.75in+0\tabcolsep}}{\textcolor[HTML]{000000}{\fontsize{11}{22}\selectfont{\global\setmainfont{Times New Roman}{A}}}} & \multicolumn{1}{>{\centering}m{\dimexpr 0.75in+0\tabcolsep}}{\textcolor[HTML]{000000}{\fontsize{11}{22}\selectfont{\global\setmainfont{Times New Roman}{1}}}} \\





\multicolumn{1}{>{\centering}m{\dimexpr 0.75in+0\tabcolsep}}{\textcolor[HTML]{000000}{\fontsize{11}{22}\selectfont{\global\setmainfont{Times New Roman}{B}}}} & \multicolumn{1}{>{\centering}m{\dimexpr 0.75in+0\tabcolsep}}{\textcolor[HTML]{000000}{\fontsize{11}{22}\selectfont{\global\setmainfont{Times New Roman}{2}}}} \\





\multicolumn{1}{>{\centering}m{\dimexpr 0.75in+0\tabcolsep}}{\textcolor[HTML]{000000}{\fontsize{11}{22}\selectfont{\global\setmainfont{Times New Roman}{C}}}} & \multicolumn{1}{>{\centering}m{\dimexpr 0.75in+0\tabcolsep}}{\textcolor[HTML]{000000}{\fontsize{11}{22}\selectfont{\global\setmainfont{Times New Roman}{3}}}} \\

\ascline{0.75pt}{000000}{1-2}


\end{longtable}

\arrayrulecolor[HTML]{000000}

\global\setlength{\arrayrulewidth}{\Oldarrayrulewidth}

\global\setlength{\tabcolsep}{\Oldtabcolsep}

\renewcommand*{\arraystretch}{1}

\subsection{Table: papaja and
apa\_table()}\label{table-papaja-and-apa_table}

Alternatively, \textbf{in .rmd files}, we can use ``wrapper'' functions
that add in a collection of tweaks for us, like the
\texttt{apa\_table()} function in \texttt{papaja}, which will knit the
table in APA formatting.

\subsection{Table: stargazer}\label{table-stargazer}

\begin{Shaded}
\begin{Highlighting}[]
\CommentTok{\# build up two regression models}
\NormalTok{mod\_1 }\OtherTok{\textless{}{-}} \FunctionTok{lm}\NormalTok{(Sepal.Length }\SpecialCharTok{\textasciitilde{}}\NormalTok{ Sepal.Width, }\AttributeTok{data =}\NormalTok{ iris)}

\NormalTok{mod\_2 }\OtherTok{\textless{}{-}} \FunctionTok{lm}\NormalTok{(Sepal.Length }\SpecialCharTok{\textasciitilde{}}\NormalTok{ Sepal.Width }\SpecialCharTok{+}\NormalTok{ Species, }\AttributeTok{data =}\NormalTok{ iris)}

\CommentTok{\# export a table with the results of these two regression models}
\FunctionTok{stargazer}\NormalTok{(mod\_1, mod\_2,}\AttributeTok{type =} \StringTok{"text"}\NormalTok{)}
\end{Highlighting}
\end{Shaded}

\begin{verbatim}

================================================================
                                Dependent variable:             
                    --------------------------------------------
                                    Sepal.Length                
                            (1)                   (2)           
----------------------------------------------------------------
Sepal.Width               -0.223                0.804***        
                          (0.155)               (0.106)         
                                                                
Speciesversicolor                               1.459***        
                                                (0.112)         
                                                                
Speciesvirginica                                1.947***        
                                                (0.100)         
                                                                
Constant                 6.526***               2.251***        
                          (0.479)               (0.370)         
                                                                
----------------------------------------------------------------
Observations                150                   150           
R2                         0.014                 0.726          
Adjusted R2                0.007                 0.720          
Residual Std. Error  0.825 (df = 148)       0.438 (df = 146)    
F Statistic         2.074 (df = 1; 148) 128.888*** (df = 3; 146)
================================================================
Note:                                *p<0.1; **p<0.05; ***p<0.01
\end{verbatim}

\begin{Shaded}
\begin{Highlighting}[]
\CommentTok{\# you can customize the table by renaming variables, adding title, omitting some statistic, adding confidence interval etc.}
\FunctionTok{stargazer}\NormalTok{(mod\_1, mod\_2,}
           \CommentTok{\# add title to the table}
           \AttributeTok{title =} \StringTok{"Regression model results"}\NormalTok{,}
           \CommentTok{\# rename dependent variable}
           \AttributeTok{dep.var.labels =} \StringTok{"The length of sepal"}\NormalTok{,}
           \CommentTok{\# rename independent variable}
           \AttributeTok{covariate.labels =} \FunctionTok{c}\NormalTok{(}\StringTok{"The width of sepal"}\NormalTok{, }\StringTok{"Versicolor"}\NormalTok{, }\StringTok{"Virginica"}\NormalTok{),}
           \CommentTok{\# "n" refers to the number of observations, "f" refers to the F stastic}
           \AttributeTok{omit.stat =} \FunctionTok{c}\NormalTok{(}\StringTok{"n"}\NormalTok{,}\StringTok{"f"}\NormalTok{),}
           \CommentTok{\# add confidence interval and set the level to 0.95}
           \AttributeTok{ci =} \ConstantTok{TRUE}\NormalTok{, }\AttributeTok{ci.level =} \FloatTok{0.95}\NormalTok{,}
           \CommentTok{\# keep only two digits}
           \AttributeTok{digits =} \DecValTok{2}\NormalTok{,}
           \CommentTok{\#display in plain text (rather than LaTeX)}
           \AttributeTok{type =} \StringTok{"text"}\NormalTok{)}
\end{Highlighting}
\end{Shaded}

\begin{verbatim}

Regression model results
===================================================
                          Dependent variable:      
                    -------------------------------
                          The length of sepal      
                          (1)             (2)      
---------------------------------------------------
The width of sepal       -0.22          0.80***    
                     (-0.53, 0.08)   (0.60, 1.01)  
                                                   
Versicolor                              1.46***    
                                     (1.24, 1.68)  
                                                   
Virginica                               1.95***    
                                     (1.75, 2.14)  
                                                   
Constant                6.53***         2.25***    
                     (5.59, 7.46)    (1.53, 2.98)  
                                                   
---------------------------------------------------
R2                       0.01            0.73      
Adjusted R2              0.01            0.72      
Residual Std. Error 0.83 (df = 148) 0.44 (df = 146)
===================================================
Note:                   *p<0.1; **p<0.05; ***p<0.01
\end{verbatim}

\begin{Shaded}
\begin{Highlighting}[]
\FunctionTok{stargazer}\NormalTok{(mod\_1, mod\_2,}
          \CommentTok{\# add title to the table}
          \AttributeTok{title =} \StringTok{"Regression model results"}\NormalTok{,}
          \CommentTok{\# View documentation to see journal styles available}
          \CommentTok{\#style = "asr",}
          \CommentTok{\# rename dependent variable}
          \AttributeTok{dep.var.labels =} \StringTok{"The length of sepal"}\NormalTok{,}
          \CommentTok{\# rename independent variable}
          \AttributeTok{covariate.labels =} \FunctionTok{c}\NormalTok{(}\StringTok{"The width of sepal"}\NormalTok{, }\StringTok{"Versicolor"}\NormalTok{, }\StringTok{"Virginica"}\NormalTok{),}
          \CommentTok{\# "n" refers to the number of observations, "f" refers to the F stastic}
          \AttributeTok{omit.stat =} \FunctionTok{c}\NormalTok{(}\StringTok{"n"}\NormalTok{,}\StringTok{"f"}\NormalTok{),}
          \CommentTok{\# add confidence interval and set the level to 0.95}
          \AttributeTok{ci =} \ConstantTok{TRUE}\NormalTok{, }\AttributeTok{ci.level =} \FloatTok{0.95}\NormalTok{,}
          \CommentTok{\# keep only two digits}
          \AttributeTok{digits =} \DecValTok{2}\NormalTok{)}
\end{Highlighting}
\end{Shaded}

\% Table created by stargazer v.5.2.3 by Marek Hlavac, Social Policy
Institute. E-mail: marek.hlavac at gmail.com \% Date and time: Fri, Feb
14, 2025 - 14:33:22

\begin{table}[!htbp] \centering 
  \caption{Regression model results} 
  \label{} 
\begin{tabular}{@{\extracolsep{5pt}}lcc} 
\\[-1.8ex]\hline 
\hline \\[-1.8ex] 
 & \multicolumn{2}{c}{\textit{Dependent variable:}} \\ 
\cline{2-3} 
\\[-1.8ex] & \multicolumn{2}{c}{The length of sepal} \\ 
\\[-1.8ex] & (1) & (2)\\ 
\hline \\[-1.8ex] 
 The width of sepal & $-$0.22 & 0.80$^{***}$ \\ 
  & ($-$0.53, 0.08) & (0.60, 1.01) \\ 
  & & \\ 
 Versicolor &  & 1.46$^{***}$ \\ 
  &  & (1.24, 1.68) \\ 
  & & \\ 
 Virginica &  & 1.95$^{***}$ \\ 
  &  & (1.75, 2.14) \\ 
  & & \\ 
 Constant & 6.53$^{***}$ & 2.25$^{***}$ \\ 
  & (5.59, 7.46) & (1.53, 2.98) \\ 
  & & \\ 
\hline \\[-1.8ex] 
R$^{2}$ & 0.01 & 0.73 \\ 
Adjusted R$^{2}$ & 0.01 & 0.72 \\ 
Residual Std. Error & 0.83 (df = 148) & 0.44 (df = 146) \\ 
\hline 
\hline \\[-1.8ex] 
\textit{Note:}  & \multicolumn{2}{r}{$^{*}$p$<$0.1; $^{**}$p$<$0.05; $^{***}$p$<$0.01} \\ 
\end{tabular} 
\end{table}

\begin{Shaded}
\begin{Highlighting}[]
\CommentTok{\# save the output in html format}
\FunctionTok{stargazer}\NormalTok{(mod\_1, }\AttributeTok{type =} \StringTok{"html"}\NormalTok{, }\AttributeTok{out =} \StringTok{"models.htm"}\NormalTok{)}
\end{Highlighting}
\end{Shaded}

\begin{verbatim}

<table style="text-align:center"><tr><td colspan="2" style="border-bottom: 1px solid black"></td></tr><tr><td style="text-align:left"></td><td><em>Dependent variable:</em></td></tr>
<tr><td></td><td colspan="1" style="border-bottom: 1px solid black"></td></tr>
<tr><td style="text-align:left"></td><td>Sepal.Length</td></tr>
<tr><td colspan="2" style="border-bottom: 1px solid black"></td></tr><tr><td style="text-align:left">Sepal.Width</td><td>-0.223</td></tr>
<tr><td style="text-align:left"></td><td>(0.155)</td></tr>
<tr><td style="text-align:left"></td><td></td></tr>
<tr><td style="text-align:left">Constant</td><td>6.526<sup>***</sup></td></tr>
<tr><td style="text-align:left"></td><td>(0.479)</td></tr>
<tr><td style="text-align:left"></td><td></td></tr>
<tr><td colspan="2" style="border-bottom: 1px solid black"></td></tr><tr><td style="text-align:left">Observations</td><td>150</td></tr>
<tr><td style="text-align:left">R<sup>2</sup></td><td>0.014</td></tr>
<tr><td style="text-align:left">Adjusted R<sup>2</sup></td><td>0.007</td></tr>
<tr><td style="text-align:left">Residual Std. Error</td><td>0.825 (df = 148)</td></tr>
<tr><td style="text-align:left">F Statistic</td><td>2.074 (df = 1; 148)</td></tr>
<tr><td colspan="2" style="border-bottom: 1px solid black"></td></tr><tr><td style="text-align:left"><em>Note:</em></td><td style="text-align:right"><sup>*</sup>p<0.1; <sup>**</sup>p<0.05; <sup>***</sup>p<0.01</td></tr>
</table>
\end{verbatim}




\end{document}
